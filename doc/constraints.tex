
\newcommand{\connectedOperandEquation}{
  \begin{equation}\textstyle\label{eq:connected-operand}
    \connectedOperand{p} \iff \operandTemporary{p} \neq \nullConnection{}
    \quad
    \forallIn{p}{\allOperands{}}
  \end{equation}
}

\newcommand{\liveStartEquation}{
\begin{equation}\textstyle\label{eq:live-start}
  \liveTemporary{t} \implies \temporaryLiveStart{t} =
  \operationIssueCycle{\operandOperation{\definer{t}}}
  \quad
  \forallIn{t}{\allTemporaries{}}
\end{equation}
}

\newcommand{\liveEndEquation}{
\begin{equation}\displaystyle\label{eq:live-end}
  \liveTemporary{t} \implies \temporaryLiveEnd{t} =
  \operator{max}\left(\;\;\;\maximum{\substack{p \in \users{t} \suchThat\\\operandTemporary{p} = t}}{\operationIssueCycle{\operandOperation{p}}}, \temporaryLiveStart{t} + \minimumLiveDuration{t}\right)
  \quad
  \forallIn{t}{\allTemporaries{}}
\end{equation}
}

\newcommand{\usersEquation}{
  \begin{equation}\textstyle\label{eq:users}
    \users{t} = \setBuilder{p \in \allOperands{}}{\use{p} \land t \in \temps{p}}
  \end{equation}
}

\newcommand{\connectedDefinerEquation}{
\begin{equation}\textstyle\label{eq:connected-definer}
  \liveTemporary{t} \iff \connectedOperand{\definer{t}}
  \quad
  \forallIn{t}{\allTemporaries{}}
\end{equation}
}

\newcommand{\definerEquation}{
  \begin{equation}\textstyle\label{eq:definer}
    \definer{t} = p \in \allOperands{} \suchThat \p{\neg \use{p} \land t \in \temps{p}}
  \end{equation}
}

\newcommand{\localOperandConnectionEquation}{
  \begin{equation}\textstyle\label{eq:local-operand-connection}
    \connectedOperand{p} \iff \activeOperation{\operandOperation{p}}
    \quad
    \forallIn{p}{\allOperands{}} \suchThat \neg \globalOperand{p}
  \end{equation}
}

\newcommand{\globalEquation}{
  \begin{equation}\textstyle\label{eq:global}
    \globalOperand{p} \iff \existsIn{q}{P} \suchThat \p{\adjacent{p}{q} \lor \adjacent{q}{p}}
    \quad
    \forallIn{p}{\allOperands{}}
  \end{equation}
}

\newcommand{\operandOperationEquation}{
  \begin{equation}\textstyle\label{eq:operand-operation}
     \operandOperation{p} = o \in \allOperations{} \suchThat p \in \operands{o}
  \end{equation}
}

\newcommand{\globalOperandConnectionEquation}{
  \begin{equation}\textstyle\label{eq:global-operand-connection}
    \begin{aligned}
      \connectedOperand{p} \iff
      \existsIn{q}{P} \suchThat
      \p{\adjacent{p}{q} \land \connectedOperand{q}}
      \quad
      \forallIn{p}{\allOperands{}} \suchThat \globalOperand{p}
    \end{aligned}
  \end{equation}
}

\newcommand{\activeInstructionEquation}{
  \begin{equation}\textstyle\label{eq:active-instruction}
    \activeOperation{o} \iff \operationInstruction{o} \neq \nullInstruction
    \quad
    \forallIn{o}{\allOperations{}}
  \end{equation}
}

\newcommand{\connectedUsersEquation}{
  \begin{equation}\textstyle\label{eq:connected-users}
    \liveTemporary{t} \iff \existsIn{p}{\users{t}} \suchThat
    \operandTemporary{p} = t
    \quad
    \forallIn{t}{\allTemporaries{}}
  \end{equation}
}

\newcommand{\preAssignmentEquation}{
\begin{equation}\textstyle\label{eq:pre-assignment}
  \temporaryRegister{\operandTemporary{p}} = \register{r}
  \quad
  \forallIn{p}{\allOperands{}} \suchThat \preAssigned{p\hspace{0.025cm}}{r}
\end{equation}
}

\newcommand{\preschedulingEquation}{
\begin{equation}\textstyle\label{eq:prescheduling}
  \activeOperation{o} \implies \operationIssueCycle{o} = \cycle{c}
  \quad
  \forallIn{o}{\allOperations{}} \suchThat \prescheduled{o}{c}
\end{equation}
and
\begin{equation}
  \activeOperation{o} \iff \cycle{c} < \operationIssueCycle{\operator{out}(\operationBlock{o})}
  \quad
  \forallIn{o}{\allOperations{}} \suchThat \prescheduled{o}{c}
\end{equation}
}

\newcommand{\bypassingEquation}{
\begin{equation}\textstyle\label{eq:bypassing}
  \bypassing{o}{\operationInstruction{o}}{p} \implies
  \operationIssueCycle{o} = \operationIssueCycle{\operandOperation{\definer{\operandTemporary{p}}}}
  \quad
  \forallIn{o}{\allOperations{}}, \,
  \forallIn{p}{\operands{o}}
\end{equation}
}

\newcommand{\adhocEquation}{
\begin{equation}\textstyle\label{eq:adhoc}
  \satisfy{e}
  \quad
  \forallIn{e}{\allAdhocConstraints{}}
\end{equation}
}

\newcommand{\satisfyEquation}{
  \begin{equation*}\textstyle\label{eq:satisfy}
    \begin{aligned}
      \satisfy{\tuple{\code{xor}, e_1, e_2}} & \iff \satisfy{e_1} \xor \satisfy{e_2}\\
      \satisfy{\tuple{\code{and}, e_1, e_2}} & \iff \satisfy{e_1} \land \satisfy{e_2}\\
      \satisfy{\tuple{\code{implies}, e_1, e_2}} & \iff \satisfy{e_1} \implies \satisfy{e_2}\\
      \satisfy{\tuple{\code{active}, o}} & \iff \activeOperation{o}\\
      \satisfy{\tuple{\code{connects}, p, t}} & \iff \operandTemporary{p} = t\\
      \satisfy{\tuple{\code{implements}, o, \code{i}}} & \iff \operationInstruction{o} = \code{i}\\
      \satisfy{\tuple{\code{distance}, d, u, n}} & \iff \operationIssueCycle{u} \ge \operationIssueCycle{d} + n\\
    \end{aligned}
  \end{equation*}
}

\newcommand{\disjointLiveRangesEquation}{
\begin{equation}\textstyle\label{eq:disjoint-live-ranges}
  \disjoint{
    \set{\tuple{
        \temporaryRegister{t},
        \temporaryRegister{t} + \width{t} \times \liveTemporary{t},
        \temporaryLiveStart{t},
        \temporaryLiveEnd{t}}
      \! \suchThat \!
      t \in \temporaries{b}}}
  \;\;
  \forallIn{b}{\allBlocks{}}
\end{equation}
}

\newcommand{\blockTemporariesEquation}{
  \begin{equation}\textstyle\label{eq:block-temporaries}
    \temporaries{b} = \setBuilder{t \in T}{\operationBlock{\operandOperation{\definer{t}}} = b}
  \end{equation}
}

\newcommand{\registerClassEquation}{
\begin{equation}\textstyle\label{eq:instruction-selection}
  \temporaryRegister{\operandTemporary{p}}
  \in
  \registerClass{o}{\operationInstruction{o}}{p}
  \quad
  \forallIn{o}{\allOperations{}}, \,
  \forallIn{p}{\operands{o}}
\end{equation}
}

\newcommand{\congruenceEquation}{
  \begin{equation}\textstyle\label{eq:congruence}
    \connectedOperand{p} \land
    \connectedOperand{q} \implies
    \temporaryRegister{\operandTemporary{p}} = \temporaryRegister{\operandTemporary{q}}
    \quad
    \forallIn{p, q}{\allOperands{}} \suchThat \adjacent{p}{q}
  \end{equation}
}

\newcommand{\alignmentEquation}{
\begin{equation}\textstyle\label{eq:alignment}
  \begin{aligned}
    \operationInstruction{\operandOperation{p}} = i \land
    \operationInstruction{\operandOperation{q}} = j \implies
    \temporaryRegister{\operandTemporary{p}} = \temporaryRegister{\operandTemporary{q}} + \alignmentDistance{p}{i}{q}{j} \\
    \forallIn{p, q}{\allOperands{}}, \,
    \forallIn{i, j}{I}
    \suchThat \alignedOperands{p}{i}{q}{j}
  \end{aligned}
\end{equation}
}

\newcommand{\packingEquation}{
\begin{equation}\textstyle\label{eq:packing}
  \begin{aligned}
    \connectedOperand{p} \land \connectedOperand{q}
    \implies
    &\temporaryRegister{\operandTemporary{q}} =
    \temporaryRegister{\operandTemporary{p}} +
    \begin{cases}
      \displaystyle{} \width{p} &
      \text{if }
      \temporaryRegister{\operandTemporary{p}}
      \eat \eat \eat \eat \mod \eat \p{\width{p} \times 2} = 0 \\
      \displaystyle{} - \width{p} &
      \text{otherwise}
    \end{cases}\\
    &\hspace{5cm}\forallIn{p, q}{\allOperands{}} \suchThat \packedOperands{p}{q}
  \end{aligned}
\end{equation}
}

\newcommand{\extensionalEquation}{
\begin{equation}\textstyle\label{eq:extensional}
  \extensional{\tuple{p,q},\extensionalTable{p}{q}}
  \quad
  \forallIn{p, q}{\allOperands{}} \suchThat \extensionalOperands{p}{q}
\end{equation}
}

\newcommand{\dataPrecedencesEquation}{
\begin{equation}\textstyle\label{eq:data-precedences}
  \begin{aligned}
    \operandTemporary{q} = t &\implies c_u \ge c_d + \latency{d}{\operationInstruction{d}}{p} + \slack{p} + \latency{u}{\operationInstruction{u}}{q} + \slack{q}\\
    & \forall t \in \allTemporaries{},\\
    & \forall p \in \set{\definer{t}}, \forall d \in \set{\operandOperation{p}}\\
    & \forall q \in \users{t},         \forall u \in \set{\operandOperation{q}}
  \end{aligned}
\end{equation}
}

\newcommand{\slackEquation}{
  \begin{equation}\textstyle\label{eq:slack}
    \slack{p} =
    \begin{cases}
      \displaystyle{} \operandLatencySlack{p} &
      \text{if }
      \globalOperand{p} \\
      \displaystyle{} 0 &
      \text{otherwise}
    \end{cases}
  \end{equation}
}

\newcommand{\fixedPrecedencesEquation}{
\begin{equation}\textstyle\label{eq:fixed-precedences}
  \activeOperation{d} \land \activeOperation{u} \implies \operationIssueCycle{u} \ge \operationIssueCycle{d} + \minimumDistance{d}{u}{\operationInstruction{d}} \quad
  \forall b \in B, \,
  \forall \sequence{d,u} \in \dependencyGraph{b}
\end{equation}
}

\newcommand{\activationEquation}{
\begin{equation}\textstyle\label{eq:activation}
  \exists{} \, o' \in \allOperations{} \suchThat \operationInstruction{o'} \in \activators{o} \implies \activeOperation{o} \quad
  \forall o \in \allOperations{}
\end{equation}
}

\newcommand{\slackBalancingEquation}{
  \begin{equation}\textstyle\label{eq:slack-balancing}
    \operandLatencySlack{p} + \operandLatencySlack{q} = 0
    \quad
    \forallIn{p, q}{\allOperands{}} \suchThat \adjacent{p}{q}
  \end{equation}
}

\newcommand{\processorResourcesEquation}{
  \begin{eqnarray}\textstyle\label{eq:processor-resources}
    \cumulative{
      \setBuilder
          {\tuple{
              \operationIssueCycle{o} +
              \offset{\operationInstruction{o}}{\processorResource{r}}, \eat
              \units{\operationInstruction{o}}{\processorResource{r}},
              \duration{\operationInstruction{o}}{\processorResource{r}}
              \eat}}
          {\eat o \in \operations{b}},
          \capacity{\processorResource{r}}}\notag{}\hspace{-0.5cm}\\
    \forallIn{b}{\allBlocks{}}, \forallIn{\processorResource{r}}{R}
  \end{eqnarray}
}

\newcommand{\blockOperationsEquation}{
  \begin{equation}\textstyle\label{eq:block-operations}
    \operations{b} = \setBuilder{o \in O}{\operationBlock{o} = b}
  \end{equation}
}

\newcommand{\genericObjectiveEquation}{
  \begin{equation*}\label{eq:generic-objective}
    \summation{b \in B}{\blockWeight{b} \times \blockCost{b}}
  \end{equation*}
}

\newcommand{\blockWeigthEquation}{
  \begin{equation}\textstyle\label{eq:block-weight}
    \blockWeight{b} =
    \begin{cases}
      \displaystyle{} \blockFrequency{b} &
      \text{if }
      \optimizeDynamic{0} \\
      \displaystyle{} 1 &
      \text{otherwise}
    \end{cases}
  \end{equation}
}

\newcommand{\blockCostEquation}{
  \begin{equation}\textstyle\label{eq:block-cost}
    \blockCost{b} =
    \begin{cases}
      \displaystyle{} \operationIssueCycle{\blockOut{b}} &
      \text{if }
      \optimizeResource{0} = \code{cycles} \\
      \textstyle{} \sum_{o \in \operations{b}}{\units{\operationInstruction{o}}{\optimizeResource{0}}} &
      \text{otherwise}
    \end{cases}
  \end{equation}
}
