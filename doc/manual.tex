\documentclass[11pt]{report}
\usepackage{graphicx}
\usepackage{abstract}
\usepackage[utf8]{inputenc}
\usepackage[english]{babel}
\usepackage[hidelinks,breaklinks]{hyperref}
\usepackage{fancyvrb}
\usepackage{longtable}
\usepackage{tcolorbox}
\tcbuselibrary{skins,breakable}
\usepackage{etoolbox}
\usepackage{amsmath}
\usepackage{float}
\usepackage{tikz}
\usetikzlibrary{positioning, fit}
\usepackage{unison}
\usepackage[left=3cm, right=3cm]{geometry}
\usepackage[adobe-utopia]{mathdesign}

\setcounter{secnumdepth}{2}

\title{The Unison Manual}

\author{Roberto Castañeda Lozano}

\date{}

\renewcommand{\abstractname}{}
\renewcommand{\absnamepos}{empty}

% Generated with https://coolors.co
% TODO: find a better color palette
\definecolor{bgcolor1}{rgb}{1, 0.9921568627, 0.5960784314}%
\definecolor{bgcolor2}{rgb}{0.7411764706, 0.8941176471, 0.6549019608}%
\definecolor{bgcolor3}{rgb}{0.7019607843, 0.8235294118, 0.6980392157}%
\definecolor{bgcolor4}{rgb}{0.6235294118, 0.7333333333, 0.8}%
\definecolor{bgcolor5}{rgb}{0.4784313725, 0.6117647059, 0.7764705882}%

\hypersetup{
  colorlinks = true,
}

\newenvironment{colorBox}[1]
 {\begin{tcolorbox}[
    breakable,
    colback=#1,
    colframe=white,
    boxrule=0pt
  ]}
 {\end{tcolorbox}}

\newenvironment{command}
 {\VerbatimEnvironment
  \begin{colorBox}{bgcolor1}
  \begin{Verbatim}[fontsize=\small]}
 {\end{Verbatim}\end{colorBox}}

\newenvironment{codeBox}
 {\begin{colorBox}{bgcolor4!20}}
 {\end{colorBox}}

\begin{document}

\maketitle

\setcounter{tocdepth}{2}
\tableofcontents

\chapter*{Introduction}
\addcontentsline{toc}{chapter}{Introduction}

You are reading the manual of \href{http://unison-code.github.io/}{Unison}: a
simple, flexible, and potentially optimal open-source tool that performs
integrated \href{https://en.wikipedia.org/wiki/Register_allocation}{register
  allocation} and
\href{https://en.wikipedia.org/wiki/Instruction_scheduling}{instruction
  scheduling} using
\href{https://en.wikipedia.org/wiki/Constraint_programming}{constraint
  programming}.

Unison can be used as an alternative or as a complement to the algorithms
applied by standard compilers such as \href{https://gcc.gnu.org/}{GCC} and
\href{http://llvm.org/}{LLVM}.
%
Unison is particularly easy to integrate with the latter as a driver is already
available (see Chapter~\ref{sec:llvm-integration} for details).

This manual is divided into two main parts: Part~\ref{part:using-unison}
(Chapters~\ref{sec:license-contact-and-acknowledgments}
to~\ref{sec:llvm-integration}) is devoted to the use of Unison, while
Part~\ref{part:developing-and-extending-unison} (Chapters~\ref{sec:architecture}
to~\ref{sec:target-description}) deals with its development and extension.

Chapter~\ref{sec:license-contact-and-acknowledgments} discusses licensing
aspects and provides contact information.
%
Chapter~\ref{sec:getting-started} contains instructions to download, build,
install, and test Unison.
%
Chapter~\ref{sec:llvm-integration} describes how to use the LLVM driver.

Chapter~\ref{sec:architecture} outlines the architecture of Unison.
%
Chapter~\ref{sec:unison-ir} describes the intermediate representation
(\emph{Unison IR}).
%
Chapter~\ref{sec:combinatorial-model} formulates the combinatorial model that
lies at the core of the Unison approach.
%
Chapter~\ref{sec:target-description} provides information about how processors
are described in Unison.

Appendix~\ref{sec:further-reading} provides references for further reading and
other sources of documentation.

\part{Using Unison}
\label{part:using-unison}

\chapter{License, Contact, and Acknowledgments}%
\label{sec:license-contact-and-acknowledgments}

Unison is developed at the \href{https://www.sics.se/}{Swedish Institute of
  Computer Science} in collaboration with \href{https://www.kth.se/en}{KTH Royal
  Institute of Technology} in Stockholm, Sweden.

Unison and the Unison Driver for LLVM are released under the BSD3 open-source
license:

\fvset{fontsize=\footnotesize}
\begin{codeBox}
\BVerbatimInput{verbatim/LICENSE}
\end{codeBox}

Unison includes code from the following projects:

\begin{itemize}
\item \href{https://hackage.haskell.org/package/Graphalyze}{Graphalyze} (in some
  graph algorithms of \code{unison} package).
  %
  The code is licensed under
  \href{https://hackage.haskell.org/package/Graphalyze/src/LICENSE}{BSD2}.
\item Erik Ekström's
  \href{https://www.sics.se/\%7ercas/teaching/ErikEkstrom_2015.pdf}{master's
    thesis} (in parts of the presolver).
  %
  The code is licensed under BSD3 but the copyright is held by Erik Ekström.

\item Mikael Almgren's
  \href{https://www.sics.se/\%7ercas/teaching/MikaelAlmgren_2015.pdf}{master's
    thesis} (in parts of the presolver).
  %
  The code is licensed under BSD3 but the copyright is held by Mikael Almgren.
\end{itemize}

The Unison Driver is built on top of the LLVM Compiler Infrastructure which is
licensed under the University of Illinois/NCSA Open Source License, see
\href{http://llvm.org/}{the LLVM website} for details.

Furthermore, Unison makes extensive use of other open-source components,
including:

\begin{itemize}
\item \href{http://www.gecode.org/}{Gecode}
\item \href{https://www.qt.io/}{Qt}
\item \href{http://www.graphviz.org/}{Graphviz}
\item \href{https://www.haskell.org/platform/}{Haskell Platform}
\item Various Haskell packages (see the \code{Build-Depends} field in
  \href{https://github.com/unison-code/unison/blob/master/src/unison/unison.cabal}{this}
  and
  \href{https://github.com/unison-code/unison/blob/master/src/unison-specsgen/unison-specsgen.cabal}{this}
  package descriptions)
\end{itemize}

For further license detail on these components, please check their websites.

Unison is designed, developed, and maintained by

\begin{itemize}
\item Roberto Castañeda Lozano (\href{mailto:rcas@sics.se}{rcas@sics.se})
\item Mats Carlsson (\href{mailto:matsc@sics.se}{matsc@sics.se})
\item Gabriel Hjort Blindell (\href{mailto:ghb@kth.se}{ghb@kth.se})
\item Christian Schulte (\href{mailto:cschulte@kth.se}{cschulte@kth.se})
\end{itemize}

Other people have also collaborated in the development of Unison:
%
\begin{itemize}
\item Özgür Akgün
\item Mikael Almgren
\item Noric Couderc
\item Frej Drejhammar
\item Erik Ekström
\item Bevin Hansson
\item Jan Tomljanović
\item Kim-Anh Tran
\end{itemize}

\chapter{Getting Started}%
\label{sec:getting-started}

Unison has to be built from source as we do not yet provide precompiled
packages.
%
The tool is known to work on Linux and it might work on other platforms such as
macOS and Windows as all software dependencies claim to be portable across
these.

Building, installing, and testing Unison in Linux is relatively straightforward.
%
Just take the following steps.

\section{Downloading}

Unison is hosted by \href{https://github.com/unison-code/unison}{GitHub}.
%
The easiest way to access its source code (including the history) is by running:
%
\begin{command}
git clone git@github.com:unison-code/unison.git
\end{command}

\section{Prerequisites}

Unison has the following direct dependencies:
%
\begin{itemize}
\item \href{http://hackage.haskell.org/platform/}{Haskell platform}
\item \href{https://www.qt.io/}{Qt (version 4.x)}
\item \href{http://www.graphviz.org/}{Graphviz library}
\item \href{http://www.gecode.org/}{Gecode (svn revision 16106 or newer)}
\end{itemize}
%
To get the first three dependencies in Debian-based distributions, just run:
%
\begin{command}
apt-get install haskell-platform libqt4-dev libgraphviz-dev
\end{command}

The source of the required Gecode revision can be fetched with:
%
\begin{command}
svn --username anonymous checkout -r16106 \
https://svn.gecode.org/svn/gecode/trunk
\end{command}

\section{Building}

Just go to the \code{src} directory and run:
%
\begin{command}
make build
\end{command}

\section{Testing}

Unison contains a test suite with a few functions where different targets and
optimization goals are exercised.
%
To execute the tests just run:
%
\begin{command}
make test
\end{command}

\section{Installing}

The building process generates three binaries.
%
The installation process consists in copying the binaries into the appropriate
system directory.
%
To install the binaries under the default directory \code{usr/local} just run:
%
\begin{command}
make install
\end{command}

The installation directory is specified by the Makefile variable \code{PREFIX}.
%
To install the binaries under an alternative directory \code{\$DIR} just run:
%
\begin{command}
make install PREFIX=$DIR
\end{command}

\section{Running}
%
This manual uses the iterative version of the factorial function as a running
example.
%
A possible C implementation is as follows:
%
\fvset{fontsize=\small}
\begin{codeBox}
\BVerbatimInput{code/factorial.c}
\end{codeBox}


When used as a standalone tool, Unison takes as input a function in LLVM's
\href{http://llvm.org/docs/MIRLangRef.html}{Machine IR format (MIR)}.
%
In this format, instructions of a certain processor have already been selected.
%
The factorial function in MIR format with
\href{https://developer.qualcomm.com/software/hexagon-dsp-sdk/dsp-processor}{Hexagon
  V4} instructions looks as follows:
%
\fvset{fontsize=\small}
\begin{codeBox}
\BVerbatimInput{code/factorial.mir}
\end{codeBox}
%
To execute Unison on this function and obtain the optimal register allocation
and instruction schedule for Hexagon V4, just run the following command from the
top of the Git repository:
%
\begin{command}
uni run doc/code/factorial.mir --goal=speed
\end{command}

This command outputs a function in MIR format where registers are allocated and
instructions are scheduled.
%
The function is thus already very close to assembly code:
%
\fvset{fontsize=\small}
\begin{codeBox}
\BVerbatimInput{code/factorial.unison.mir}
\end{codeBox}

The \code{uni} tool has several options and commands such as \code{run}.
%
Detailed information about each option and command can be obtained by running:
%
\begin{command}
uni --help
\end{command}

Unison can be used as a standalone tool as illustrated above but is only really
useful as a complement to a full-fledged compiler.
%
The next section gives instructions to use Unison together with LLVM.

\chapter{LLVM Integration}%
\label{sec:llvm-integration}

Unison is accompanied with a driver that allows transparent integration with the
LLVM compiler infrastructure.
%
In particular, the driver enables
\href{http://llvm.org/docs/CommandGuide/llc.html}{LLVM's code generator}
(\code{llc}) to run Unison transparently instead of its standard register
allocation and instruction scheduling algorithms.
%
Figure~\ref{fig:llvm} shows how the LLVM driver interfaces with Unison to
produce assembly code all the way from source code.
%
Arcs between components are labeled with the file extension corresponding to the
shared data file.

\begin{figure}[H]
  \centering
  \scalebox{0.75}{\def\lightGray{gray!15}%
\newdimen\zerolinewidth%
\newdimen\cornerRadius%
\newdimen\lessCornerRadius%
\cornerRadius=20pt%
\lessCornerRadius=10pt%
\tikzset{%
  zero line width/.code={%
    \zerolinewidth=\pgflinewidth
    \tikzset{line width=0cm}%
  },
  use line width/.code={%
    \tikzset{line width=\the\zerolinewidth}%
  },
  file/.style={%
    outer sep=0,
  },
  input/.style={%
    file,
  },
  output/.style={%
    file,
  },
  box/.style={%
    draw,
    fill=white,
  },
  flow/.style={%
    ->,
    >=stealth,
  },
  unison/.style={%
    draw,
    inner xsep=3mm,
    inner ysep=4mm,
    dash pattern=on 2mm off 2mm,
    fill=bgcolor2
  },
  unison stage/.style={%
    box,
    fill=bgcolor1
  },
}%
\pgfdeclarelayer{layer0}%
\pgfdeclarelayer{layer1}%
\pgfdeclarelayer{layer2}%
\pgfdeclarelayer{layer3}%
\pgfdeclarelayer{layer4}%
\pgfdeclarelayer{layer5}%
%
\pgfsetlayers{layer1,layer2}%
\begin{tikzpicture}
  \begin{pgfonlayer}{layer2}
    \node [input] (input) {};
    \node [unison stage, right=of input] (clang) {\code{clang}};
    \begin{scope}[node distance=1cm]
    \node [unison stage, right=of clang] (opt) {\code{opt}};
    \node [unison stage, right=of opt] (llc) {\code{llc}};
    \node [unison stage, below=of llc] (unison) {\code{unison}};
    \end{scope}
    \node [output, right=of llc] (output) {};
  \end{pgfonlayer}

  \begin{pgfonlayer}{layer1}
    \begin{scope}[flow]
      \draw (input) -- node[yshift=0.4cm] {\code{.c}} (clang);
      \draw (clang) -- node[yshift=0.4cm] {\code{.ll}} (opt);
      \draw (opt) -- node[yshift=0.4cm] {\code{.ll}} (llc);
      \draw ([xshift=-0.1cm] llc.south) -- node[xshift=-0.68cm, yshift=0.2cm] {\code{.mir}} ([xshift=-0.1cm] unison.north);
      \draw ([xshift=-0.2cm] llc.south) -- node[xshift=-1cm, yshift=-0.2cm] {\code{.asm.mir}} ([xshift=-0.2cm] unison.north);
      \draw ([xshift=0.2cm] unison.north) -- node[xshift=1.3cm] {\code{.unison.mir}} ([xshift=0.2cm] llc.south);
      \draw (llc) -- node[yshift=0.4cm] {\code{.s}} (output);
    \end{scope}
  \end{pgfonlayer}
\end{tikzpicture}
}
  \caption{design of the LLVM driver}
  \label{fig:llvm}
\end{figure}

Unison uses LLVM's \href{http://llvm.org/docs/MIRLangRef.html}{Machine IR format
  (MIR)} as the interface language.
%
Unison takes as input a function in MIR format (\code{.mir}) and the function
where LLVM has already performed register allocation and instruction scheduling
(\code{.asm.mir}) as a starting point for the optimization algorithm.
%
For our running example, the starting solution \code{factorial.asm.mir} looks as
follows:
%
\fvset{fontsize=\small}
\begin{codeBox}
\BVerbatimInput{code/factorial.asm.mir}
\end{codeBox}
%
The result of running Unison (\code{.unison.mir}) is sent back to \code{llc}
where the final assembly code is emitted.

As for the core Unison tool, the driver must be built from source as we do not
yet provide precompiled packages.
%
The driver is known to work on Linux and should work in all other main platforms
provided that Unison itself can be built successfully.

\section{Downloading}

The Unison driver for LLVM is hosted as a LLVM form by
\href{https://github.com/unison-code/llvm}{GitHub}.
%
The easiest way to access its source code (including the history) is by running:
%
\begin{command}
git clone git@github.com:unison-code/llvm.git
\end{command}

LLVM 3.8 is the latest LLVM supported version.
%
To access the driver for this version run the following command on the cloned
repository:
%
\begin{command}
git checkout release_38-unison
\end{command}

\section{Prerequisites}

The LLVM driver depends on Unison being installed successfully.
%
Check the \href{http://llvm.org/docs/GettingStarted.html#requirements}{LLVM
  website} for the prerequisites to build LLVM itself.

\section{Building, Testing, and Installing}

Just follow the instructions provided at
\href{http://llvm.org/docs/GettingStarted.html#compiling-the-llvm-suite-source-code}{LLVM's
  website} as usual.

\section{Running with llc}

To execute \code{llc} such that Unison is used for register allocation and
instruction scheduling, just run the following command from the top of the core
Unison Git repository:
%
\begin{command}
llc doc/code/factorial.ll -march=hexagon -mcpu=hexagonv4 -unison
\end{command}

Currently, Unison supports the following LLVM targets (defined by
\code{march}-\code{mcpu}-\code{mattr} triples):

\begin{center}
  \begin{tabular}{l l l l}
    target & \code{-march=} & \code{-mcpu=} & \code{-mattr=} \\\hline
    Hexagon V4 & \code{hexagon} & \code{hexagonv4} & \\
    ARM1156T2F-S & \code{arm} & \code{arm1156t2f-s} & \code{+thumb-mode} \\
    MIPS32 & \code{mips} & \code{mips32} & \\
  \end{tabular}
\end{center}

Other flags (with a \code{-unison} prefix) can be used to control the execution
of Unison, run \code{llc --help} for details.

\section{Running with clang}

To execute \code{clang} with Unison's register allocation and instruction
scheduling for Hexagon V4, first build and install a matching \code{clang}
version (see the \href{http://clang.llvm.org/get_started.html}{Clang website}
for details).
%
Then just run the following command from the top of the core Unison Git
repository:
%
\begin{command}
clang doc/code/factorial.c -target hexagon -mllvm -unison
\end{command}

Alternatively, functions can be annotated with the \code{"unison"} attribute to
indicate that Unison should be run on them:
%
\begin{command}
__attribute__((annotate("unison")))
\end{command}

\part{Developing and Extending Unison}
\label{part:developing-and-extending-unison}

\chapter{Architecture}%
\label{sec:architecture}

As usual in compiler construction, Unison is organized as a chain of
transformation components through which the program flows.
%
Each intermediate representation of the program is stored in a file.
%
Unison takes a mandatory and an optional file as input.
%
The mandatory file is a function in LLVM's
\href{http://llvm.org/docs/MIRLangRef.html}{Machine IR format} (\code{.mir})
where:
%
\begin{itemize}
\item instructions of a certain processor have already been selected,
\item the code is in
  \href{https://en.wikipedia.org/wiki/Static_single_assignment_form}{Static
    Single Assignment} (SSA) form,
\item instructions use and define temporaries (program variables at the code
  generation level) rather than processor registers, and
\item instructions are not yet scheduled.
\end{itemize}
%
The optional file (\code{.asm.mir}) is also in Machine IR format and represents
the same function where register allocation and instruction scheduling has
already been applied by an external tool (typically \code{llc}).
%
This file contains as a structured representation of assembly code that Unison
can take as a starting solution within the optimization process.
%
Unison generates as output a single function (\code{.unison.mir}) with the given
function after register allocation and instruction scheduling.
%
If Unison cannot improve the initial solution provided in the \code{.asm.mir}
file, this is just shown as output.

Figure~\ref{fig:components} shows the main components involved in compilation of
intermediate (\code{.mir}) code to assembly (\code{.unison.mir}) code.
%
Arcs between components are labeled with the file extension corresponding to the
shared data file.
%
The Unison components are enclosed by a dashed rectangle.

\begin{figure}[H]
  \centering
  \scalebox{0.65}{\def\lightGray{gray!15}%
\newdimen\zerolinewidth%
\newdimen\cornerRadius%
\newdimen\lessCornerRadius%
\cornerRadius=20pt%
\lessCornerRadius=10pt%
\tikzset{%
  zero line width/.code={%
    \zerolinewidth=\pgflinewidth
    \tikzset{line width=0cm}%
  },
  use line width/.code={%
    \tikzset{line width=\the\zerolinewidth}%
  },
  file/.style={%
    outer sep=0,
  },
  input/.style={%
    file,
  },
  output/.style={%
    file,
  },
  box/.style={%
    draw,
    fill=white,
  },
  flow/.style={%
    ->,
    >=stealth,
  },
  unison/.style={%
    draw,
    inner xsep=3mm,
    inner ysep=4mm,
    dash pattern=on 2mm off 2mm,
    fill=bgcolor2
  },
  unison stage/.style={%
    box,
    fill=bgcolor1
  },
}%
\pgfdeclarelayer{layer0}%
\pgfdeclarelayer{layer1}%
\pgfdeclarelayer{layer2}%
\pgfdeclarelayer{layer3}%
\pgfdeclarelayer{layer4}%
\pgfdeclarelayer{layer5}%
%
\pgfsetlayers{layer1,layer2}%
\begin{tikzpicture}
  \newcommand{\twoLines}[2]{\begin{tabular}{c}\code{#1}\\\code{#2}\end{tabular}}
  \newcommand{\uniCommand}[1]{\twoLines{uni}{#1}}
  \begin{pgfonlayer}{layer2}
    \node [input] (input) {};
    \node [unison stage, right=of input] (importer) {\uniCommand{import}};
    \begin{scope}[node distance=0.6cm]
    \node [unison stage, right=of importer] (linearizer) {\uniCommand{linearize}};
    \node [unison stage, right=of linearizer] (extender) {\uniCommand{extend}};
    \node [unison stage, right=of extender] (augmenter) {\uniCommand{augment}};
    \node [unison stage, right=of augmenter] (modeler) {\uniCommand{model}};
    \node [input, above=1cm of modeler] (baseinput) {};
    \node [unison stage, right=of modeler] (presolver) {\twoLines{gecode-}{presolver}};
    \node [unison stage, right=of presolver] (solver) {\twoLines{gecode-}{solver}};
    \node [unison stage, right=of solver] (exporter) {\uniCommand{export}};
    \end{scope}
    \node [output, right=of exporter] (output) {};
  \end{pgfonlayer}

  \coordinate (augmenterbot) at ([yshift=-0.8cm] augmenter.south);
  \coordinate (exporterbot)  at ([yshift=-0.8cm] exporter.south);

  \begin{pgfonlayer}{layer1}
    \node [unison, fit=(importer) (linearizer) (extender) (augmenter) (augmenterbot) (modeler) (presolver) (solver) (exporter)] (unison) {};
  \end{pgfonlayer}

  \begin{pgfonlayer}{layer1}
    \begin{scope}[flow]
      \draw (input) -- node[xshift=-0.5cm, yshift=-0.4cm] {\code{.mir}} (importer);
      \draw (importer) -- node[yshift=-0.9cm] {\code{.uni}} (linearizer);
      \draw (linearizer) -- node[yshift=-0.9cm] {\code{.lssa.uni}} (extender);
      \draw (extender) -- node[yshift=-0.9cm] {\code{.ext.uni}} (augmenter);
      \draw (augmenter) -- node[yshift=-0.9cm] {\code{.alt.uni}} (modeler);
      \draw (baseinput) -- node[auto, yshift=0.3cm] {\code{.asm.mir}} (modeler);
      \draw (modeler) -- node[yshift=-0.9cm] {\code{.json}} (presolver);
      \draw (presolver) -- node[yshift=-0.9cm] {\code{.ext.json}} (solver);
      \draw (solver) -- node[yshift=-0.9cm] {\code{.out.json}} (exporter);
      \draw (exporter) -- node[xshift=1.2cm, yshift=-0.4cm] {\code{.unison.mir}} (output);
    \end{scope}
  \draw (augmenter) -- (augmenterbot);
  \draw (augmenterbot) -- (exporterbot);
  \draw [flow] (exporterbot) -- (exporter);
  \end{pgfonlayer}

\end{tikzpicture}
}
  \caption{main components and boundaries of Unison}
  \label{fig:components}
\end{figure}

The function of each component is:

\begin{description}

\item [\code{uni import}:] transform the instruction-selected program into
  Unison IR;

\item [\code{uni linearize}:] transform the program to Linear Static Single
  Assignment form;

\item [\code{uni extend}:] extend the program with copies;

\item [\code{uni augment}:] augment the program with alternative temporaries;

\item [\code{uni model}:] formulate a combinatorial problem combining global
  register allocation and instruction scheduling;

\item [\code{gecode-presolver}:] produce an equivalent combinatorial problem
  that is easier to solve;

\item [\code{gecode-solver}:] solve the combinatorial problem;

\item [\code{uni export}:] generate the almost-assembly program with the
  solution to the combinatorial problem.

\end{description}

Chapter~\ref{sec:unison-ir} gives further detail on the Unison IR (\code{.uni})
and the different transformations that are applied to it.

\chapter{Unison IR}
\label{sec:unison-ir}

The intermediate representation (IR) used in Unison is simply referred to as
\emph{Unison IR}.
%
Unison IR is a low-level,
\href{https://en.wikipedia.org/wiki/Control_flow_graph}{control-flow
  graph}-based IR (just like LLVM's Machine IR) that exposes the structure of
the program and the multiple register allocation and instruction scheduling
decisions to be formulated in the combinatorial model.
%
Unison IR has the following distinguishing features:
%
\begin{description}
\item [linear static single assignment form (LSSA)] LSSA is a program form in
  which temporaries (program variables at Unison IR's level) are local to
  \href{https://en.wikipedia.org/wiki/Basic_block}{basic blocks} (\emph{blocks}
  for short) and relations across temporaries from different blocks are made
  explicit.
\item [optional copies] Unison IR includes optional copy operations that can be
  deactivated or implemented by alternative instructions.
\item [alternative temporaries] Unison IR allows operations to use alternative
  temporaries that hold the same value.
\end{description}

Unison IR as required for the combinatorial model formulation is constructed
incrementally in four transformations as shown in Figure~\ref{fig:components}.
%
This section introduces the elements of Unison IR progressively by following the
transformation chain for the running example.

\section{Initial Unison IR}

The initial Unison IR (\code{.uni}) after running the \code{uni import}
component has a very similar structure to the input MIR function (\code{.mir}).
%
This is how \code{factorial.uni} looks like:

\fvset{fontsize=\small}
\begin{codeBox}
\BVerbatimInput{code/factorial.uni}
\end{codeBox}

A Unison IR file consists of multiple sections:

\begin{description}
\item [\code{function}:] name of the function.
\item [code:] actual code of the function (more details are given below).
\item [\code{adjacent}:] relations among temporaries from different blocks (more
  details are given in Section~\ref{sec:lssa}).
\item [\code{fixed-frame}:] frame objects (such as arguments passed by the
  \href{https://en.wikipedia.org/wiki/Call_stack}{stack}) with a fixed position.
\item [\code{frame}:] frame objects with a variable position.
\item [\code{stack-pointer-offset}:] offset to add to the positions of the frame
  objects in the stack frame.
\item [\code{jump-table}:] blocks to which
  \href{https://en.wikipedia.org/wiki/Branch_table}{jump tables} in the code can
  jump.
\item [\code{goal}:] optimization goal for Unison (either \code{speed} or
  \code{size}).
\item [\code{source}:] source code from which the function originates (typically
  LLVM IR).
\end{description}

The code section consists of a list of blocks.
%
Each block has some attributes (whether it is the entry or exit point of the
function, whether it returns to its caller function, and its estimated or
profiled execution frequency) and a list of operations.
%
An operation consists of an identifier (for example \code{o8}), a list of
definition operands (\code{[t6]}), an instruction that implements the operation
(\code{A2\_addi}), and a list of use operands (\code{[t5, -1]}).
%
Definition operands are always temporaries (\code{t6}); use operands can be
temporaries (\code{t5}), block references (\code{b2} in \code{o3}, or Machine IR
operands (\code{-1} in \code{o8}) that are only passed through by Unison.

Some Unison IR operations are virtual, that is, they contribute to the
definition of the function semantics but do not appear in Unison's output.
%
An example is the \code{in} and \code{out} delimiter operations, which define
(use) temporaries that are
\href{https://en.wikipedia.org/wiki/Live_variable_analysis}{live-in} (out) at
the entry (exit) point of each block.
%
Another example is the \code{phi} operations in blocks \code{b1} and \code{b2}
(the initial Unison IR is in
\href{https://en.wikipedia.org/wiki/Static_single_assignment_form}{Static Single
  Assignment} (SSA) form).
%
A \code{phi} operation (for example \code{o14}) at the beginning of a block
defines a temporary with the value of different temporaries depending on the
preceding block that is executed (for example \code{t9} is set to the value of
\code{t2} if \code{b0} is the preceding block or \code{t7} if \code{b1} is the
preceding block).

In the Unison IR, a preassignment of a temporary \code{t} to a register \code{r}
at a certain point is represented by \code{t:r}.
%
For example, the calling convention in Hexagon dictates that the first argument
of a function is passed in register \code{r0}, which is modeled by the
definition operands \code{t0:r0} in the \code{(in)} delimiter of the entry block
\code{b0}.

\section{Linearized Unison IR}\label{sec:lssa}

The linearized Unison IR (\code{.lssa.uni}) after running the \code{uni
  linearize} component is in Linear Static Single Assignment (LSSA) form.
%
This is how \code{factorial.lssa.uni} looks like:

\fvset{fontsize=\small}
\begin{codeBox}
\BVerbatimInput{code/factorial.lssa.uni}
\end{codeBox}

The difference with the initial Unison IR is that each temporary \code{t} (for
example \code{t1} in \code{factorial.uni}) is decomposed into one temporary per
block where \code{t} is live (for example \code{t1} in \code{b0}, \code{t6} in
\code{b1}, and \code{t11} in \code{b2}).
%
The congruence among LSSA temporaries that originate from the same temporary is
kept in the \code{adjacent} section.
%
The fact that a temporary \code{t} in a block is congruent to another temporary
\code{t'} in an immediate successor block is represented by \code{t -> t'} (for
example \code{t1 -> t6} indicates that \code{t1} in \code{b0} is congruent to
\code{t6} in the immediate successor block \code{b1}).

In the linearized Unison IR, liveness information is made explicit by defining
and using the decomposed temporaries in \code{(out)} and \code{(in)} delimiter
operations.
%
In this form, \code{(phi)} operations are no longer necessary as the information
that they convey is captured by the new elements in the IR.
%
For example, the relation between \code{t9}, \code{t2}, and \code{t7} in
\code{factorial.uni} is captured by \code{t2 -> t10} and \code{t8 -> t10} in
\code{factorial.lssa.uni} (\code{t2}, \code{t7}, and \code{t9} in
\code{factorial.uni} correspond to \code{t2}, \code{t8}, and \code{t10} in
\code{factorial.lssa.uni}).

\section{Extended Unison IR}

The transformation performed by \code{uni extend} (\code{.ext.uni}) extends the
IR with optional copy operations that can be deactivated by the solver or used
to support register allocation decisions such as spilling and live-range
splitting.
%
This is how \code{factorial.ext.uni} looks like:

\fvset{fontsize=\scriptsize}
\begin{codeBox}
\BVerbatimInput{code/factorial.ext.uni}
\end{codeBox}

A copy operation can be discarded or implemented by alternative instructions.
%
For example, \code{o15} can be deactivated if implemented by the special
\emph{null} instruction (\code{-}), or implemented by a register-to-register
move (\code{MVW}), a regular store (\code{STW}), or a zero-read-latency store
(\code{STW\_nv}).
%
The particular strategy to extend functions with copies is processor-specific
(see Chapter~\ref{sec:target-description}), but it is common to insert a copy
including store instructions after each temporary definition and a copy
including load instructions after each temporary use.
%
For example, \code{o15} and \code{o22} are inserted after the definition and
before a use of \code{t25} and \code{t23} which correspond to \code{t7} in
\code{factorial.lssa.uni}.

\section{Augmented Unison IR}

The augmented IR (\code{.alt.uni}) allows operations to use alternative
temporaries that hold the same value.
%
This is how factorial.alt.uni looks like:

\fvset{fontsize=\tiny}
\begin{codeBox}
\BVerbatimInput{code/factorial.alt.uni}
\end{codeBox}

The main change in the augmented Unison IR is the introduction of operand
identifiers (for example \code{p47}) and temporaries that can be
\emph{connected} to them (for example \code{\{-, t12, t15, ...\}}) where the
special \emph{null} connection (\code{-}) indicates that the operand is not
connected to any temporary because its operation is inactive.

Another difference is the annotation of operations with side effects.
%
A side effect reads or writes an abstract object (for example, \code{control}
for control flow or \code{pc} for Hexagon's program counter).
%
Multiple reads and writes to the same object cause dependencies among the
operations (for example, \code{o2} must be issued before \code{o8} provided the
latter is active as \code{o2} reads and \code{o8} writes the \code{control}
object.

Finally, a Hexagon specific transformation is performed where an alternative way
of implementing compare-and-jump operations (\code{o5} and \code{o8}; \code{o21}
and \code{o24}) is introduced (see the Hexagon Programmer's Reference Manual or
the comments in LLVM's \code{HexagonNewValueJump.cpp} file for further detail).

\chapter{Combinatorial Model}\label{sec:combinatorial-model}

This chapter formulates the combinatorial model of register allocation and
instruction scheduling that is at the core of Unison.
%
The combinatorial model consists of parameters (Section~\ref{sec:parameters})
describing the input program, processor, and objective; variables
(Section~\ref{sec:variables}) capturing the different decisions involved in
register allocation and instruction scheduling; constraints
(Section~\ref{sec:constraints}) relating and limiting the decisions; and an
objective function (Section~\ref{sec:objective}) to optimize for.
%
This chapter provides a raw but formal description of the model, for further
explanations please consult~\cite{Castaneda2014c}.

\section{Parameters}\label{sec:parameters}

This section lists the parameters of the combinatorial model with examples from
\code{factorial.json} (see Figure~\ref{fig:components}).

\booltrue{showJson}

\subsection{Program}

\begin{longtable}{ l p{12cm} }

  \parameter{B, O, P, T}{sets of blocks, operations, operands and temporaries}
  \json{B}{[0, 1, 2]}
  \json{O}{[0, 1, 2, \dots, 33, 34]}
  \json{P}{[0, 1, 2, \dots, 67, 68]}
  \json{T}{[0, 1, 2, \dots, 31, 32]}

  \tableSpace

  \parameter{\operationBlock{o}}{block to which operation $o$ belongs}
  \json{block}{[0, 0, \dots, 1, 1, 1, 2, 2, 2, 2, 2]}

  \tableSpace

  \parameter{\operands{o}}{set of operands of operation $o$}
  \json{operands}{[[0, 1], [2, 3], [4], \dots, [65, 66], [67], [68]]}

  \tableSpace

  \parameter{\temps{p}}{set of temporaries that can be connected to operand $p$}
  \json{temps}{[[0], [1], [-1, 0], \dots, [30], [29, 31, 32]]}

  \tableSpace

  \parameter{\use{p}}{whether $p$ is a use operand}
  \json{use}{[false, false, true, \dots, true, true]}

  \tableSpace

  \parameter{\adjacent{p}{q}}{whether operands $p$ and $q$ are adjacent}
  \json{adjacent}{[[20, 24], [21, 25], \dots, [60, 61]]}

  \tableSpace

  \parameter
  {\preAssigned{p\hspace{0.025cm}}{r}}
  {whether operand $p$ is preassigned to register $\register{r}$}
  \json{preassign}{[[0, 0], [1, 31], [68, 0]]}

  \tableSpace

  \parameter{\width{t}}{number of register atoms that temporary $t$ occupies}
  \json{width}{[1, 1, 1, \dots, 1, 1]}

  \tableSpace

  \parameter{\frequency{b}}{estimated execution frequency of block $b$}
  \json{freq}{[4, 85, 4]}

  \tableSpace

  % Below parameters are not in LCTES2014 paper

  \parameter{\alignedOperands{p}{i}{q}{j}}{whether operands $p$ and $q$ are
    aligned when implemented by instructions $i$ and $j$}
  \json{aligned}{[[69, 17, 71, 17], [70, 24, 71, 24]]}
  \jsonComment{note: the example JSON arrays are extracted from a different
    program (\code{factorial} does not contain alignment constraints)}

  \tableSpace

  \parameter{\alignmentDistance{p}{i}{q}{j}}{alignment distance of operands $p$
    and $q$ when implemented by instructions $i$ and $j$}
  \json{adist}{[0, 1, 1]}
  \jsonComment{note: this parameter is encoded with the same structure as
    \code{aligned}: each aligned operand tuple and its corresponding alignment
    distance are found in the same positions of their respective JSON arrays
    (example: \code{aligned[1]~=~[70, 24, 71, 24]}, \code{adist[1]~=~1}.)}

  \tableSpace

  \parameter{\packedOperands{p}{q}}{whether operands $p$ and $q$ are
    packed}
  \json{packed}{[[13, 14], [34, 35], [54, 55]]}
  \jsonComment{note: the example JSON arrays are extracted from a different
    program (\code{factorial} does not contain packing constraints)}

  \tableSpace

  \parameter{\minimumLiveDuration{t}}{minimum live duration of temporary $t$ if
    it is live}
  \json{minlive}{[1, 1, 1, \dots, 1, 1]}

  \tableSpace

  \parameter
  {\dependencyGraph{b}}
  {fixed dependency graph of the operations of block $b$}
  \json{dep}{[[[0, 1], \dots, [1, 8], \dots, [10, 11]], [\dots], [\dots]]}

  \tableSpace

  \parameter{\activators{o}}{set of instructions that activate operation $o$}
  \json{activators}{[[], [10, 17, 13, 19], [10, 17, 13, 19], \dots, []]}
  \jsonComment{note: the example JSON array is extracted from a different
    program (Hexagon programs do not yet yield activation constraints)}

  \tableSpace

  \parameter
  {\participative{o}{c}}
    {whether operation $o$ is \emph{participative} with cycle $\cycle{c}$}
  \json{part}{[[2, 3], [1, 1], [60, 24]]}

\end{longtable}


\subsection{Processor}

\begin{longtable}{ l p{12cm} }

  \parameter{I, R}{sets of instructions and resources}
  \json{I}{[0, 1, 2, 3, \dots, 16]}
  \json{R}{[0, 1, 2, 3, \dots, 8]}

  \tableSpace

  \parameter{\minimumDistance{o_1}{o_2}{\instruction{i}}}{min.~issue distance of
    ops.~$o_1$ and $o_2$ when $o_1$ is implemented by $\instruction{i}$}
  \json{dist}{[[[1], \dots, [0, 0, 0], \dots, [1]], [\dots], [\dots]]}
  \jsonComment{note: this parameter is encoded with the same structure as
    \code{dep}: each dependency and its corresponding distance array are found
    in the same positions of their respective JSON arrays (example:
    \code{dep[0][2]~=~[0, 3]}, \code{dist[0][2] = [1]}).}

  \tableSpace

  \parameter{\registerClass{o}{\instruction{i}}{p}}{register class in which
    operation $o$ implemented by $\instruction{i}$ accesses $p$}
  \json{class}{[[0, 0]], [[0, 0], [1, 1], [1, 9]], \dots, [[0]]}

  \tableSpace

  \parameter{\atoms{rc}}{atoms of register class $rc$}
  \json{atoms}{[[0, 1, 2, \dots, 76], \dots, [37, 39, 41, \dots, 75]]}

  \tableSpace

  \parameter{\instructions{o}}{set of instructions that can implement operation
    $o$}
  \json{instructions}{[[2], [0, 3, 4], [5], \dots, [16], [2]]}

  \tableSpace

  \parameter{\latency{o}{\instruction{i}}{p}}{latency of $p$ when its operation $o$ is implemented by $\instruction{i}$}
  \json{lat}{[[[1, 1]], [[0, 0], [0, 1], [0, 1]], \dots, [[0]]]}

  \tableSpace

  \parameter{\bypassing{o}{\instruction{i}}{p}}{whether $p$ is bypassing when its operation $o$ is implemented by $\instruction{i}$}
  \json{bypass}{[[[false, false]], [[false, false], \dots, [[false]]]}

  \tableSpace

  \parameter{\capacity{\code{r}}}{capacity of processor resource $r$}
  \json{cap}{[4, 4, 2, 1, 2, 1, 1, 2, 1]}

  \tableSpace

  \parameter{\units{\instruction{i}}{\code{r}}}
  {consumption of processor resource $r$ by instruction $\instruction{i}$}
  \json{con}{[[0, 0, \dots, 0, 0], \dots, [1, 1, 0, 0, 1, 1, 0, 0, 0]]}

  \tableSpace

  \parameter{\duration{\instruction{i}}{\code{r}}}{duration of usage of processor
    resource $r$ by instruction $\instruction{i}$}
  \json{dur}{[[0, 0, \dots, 0, 0], \dots, [1, 1, 0, 0, 1, 1, 0, 0, 0]]}

  \tableSpace

  \parameter{\offset{\instruction{i}}{\code{r}}}{offset of usage of processor
    resource $r$ by instruction $\instruction{i}$}
  \json{off}{[[0, 0, \dots, 0, 0], \dots, [0, 0, 0, 0, 0, 0, 0, 0, 0]]}

  % Below parameters are not in LCTES2014 paper

  \parameter{\alignedOperands{p}{i}{q}{j}}{whether operands $p$ and $q$ are
    aligned when implemented by instructions $i$ and $j$}
  \json{aligned}{[[69, 17, 71, 17], [70, 24, 71, 24]]}
  \jsonComment{note: the example JSON arrays are extracted from a different
    program (Hexagon programs do not yield alignment constraints)}

  \tableSpace

  \parameter{\alignmentDistance{p}{i}{q}{j}}{alignment distance of operands $p$
    and $q$ when implemented by instructions $i$ and $j$}
  \json{adist}{[0, 1, 1]}
  \jsonComment{note: this parameter is encoded with the same structure as
    \code{aligned}: each aligned operand tuple and its corresponding alignment
    distance are found in the same positions of their respective JSON arrays
    (example: \code{aligned[1]~=~[70, 24, 71, 24]}, \code{adist[1]~=~1}.)}

  \tableSpace

  \parameter{\packedOperands{p}{q}}{whether operands $p$ and $q$ are
    packed}
  \json{packed}{[[13, 14], [34, 35], [54, 55]]}
  \jsonComment{note: the example JSON arrays are extracted from a different
    program (Hexagon programs do not yield packing constraints)}

  \tableSpace

  \parameter{\extensionalOperands{p}{q}}{whether operands $p$ and $q$ are
    related extensionally}
  \json{exrelated}{[[4, 5]]}
  \jsonComment{note: the example JSON arrays are extracted from a different
    program (Hexagon programs do not yield extensional constraints)}

  \tableSpace

  \parameter{\extensionalTable{p}{q}}{table of register assignments for
    operands $p$ and $q$}
  \json{table}{[[0, 1],[2, 3],[4, 5],[6, 7]]}
  \jsonComment{note: the example JSON arrays are extracted from a different
    program (Hexagon programs do not yield extensional constraints)}

  \tableSpace

  \parameter{\activators{o}}{set of instructions that activate operation $o$}
  \json{activators}{[[], [10, 17, 13, 19], [10, 17, 13, 19], \dots, []]}
  \jsonComment{note: the example JSON array is extracted from a different
    program (Hexagon programs do not yet yield activation constraints)}

  \tableSpace

  \parameter{\callerSavedAtoms{}}{caller-saved atoms}
  \json{callersaved}{[0, 1, 2, 3, 4, 5, 6, 7, 8, 9, 10, 11, 12, 13, 14, 15, 28, 32, 33, 34, 35]}

  \tableSpace

  \parameter{\allAdhocConstraints{}}{set of ad hoc processor constraints}
  \json{E}{[[2, [1, [5, 13], [5, 14]], [5, 15]], \dots]}
  \jsonComment{note: constraints are encoded as trees of expression tuples,
    where the first element of each expression tuple encodes its type as
    follows:}
  \jsonComment{%
    \mbox{%
      \begin{tabular}{|l l|}
        \hline
        \begin{tabular}{ l l }
          \code{0} & \code{or}\\
          \code{1} & \code{and}\\
          \code{2} & \code{xor}\\
          \code{3} & \code{implies}\\
          \code{4} & \code{not}\\
          \code{5} & \code{active}\\
          \code{6} & \code{connects}\\
        \end{tabular}
        &
        \begin{tabular}{ l l }
          \code{7} & \code{implements}\\
          \code{8} & \code{distance}\\
          \code{9} & \code{share}\\
          \code{10} & \code{operand-overlap}\\
          \code{11} & \code{temporary-overlap}\\
          \code{12} & \code{caller-saved}\\
          \code{13} & \code{allocated}\\
        \end{tabular}\\
        \hline
      \end{tabular}
    }
  }
  \jsonComment{note: the example JSON array is extracted from a different
    program (Hexagon programs do not yet yield ad hoc constraints)}

\end{longtable}


\subsection{Objective}

\begin{tabular}{ l p{9.7cm} }

  \parameter{\optimizeDynamic{}}{whether to use block frequencies as weight}
  \json{optimize\_dynamic}{true}

  \tableSpace

  \parameter{\optimizeResource{}}{resource whose consumption is to be optimized}
  \json{optimize\_resource}{-1}
  \jsonComment{Note: the estimated number of cycles is encoded as resource
    \code{-1}, otherwise the usual resource numbers in \code{R} are used}

  \tableSpace

  \parameter{\maxCost{}}{upper bound of the objective}
  \json{maxf}{274}

\end{tabular}


\section{Variables}\label{sec:variables}

\begin{tabular}{ l p{12cm} }
  \modelVariable{\activeOperation{o}}{\set{0, 1}}{whether operation $o$ is active}
  \tableSpace
  \modelVariable{\operationInstruction{o}}{\instructions{o}}{instruction that implements operation $o$}
  \tableSpace
  \modelVariable{\liveTemporary{t}}{\set{0, 1}}{whether temporary $t$ is live}
  \tableSpace
  \modelVariable{\temporaryRegister{t}}{\naturalNumbersZero}{register to which temporary $t$ is assigned}
  \tableSpace
  \modelVariable{\connectedOperand{p}}{\set{0, 1}}{whether operand $p$ is connected}
  \tableSpace
  \modelVariable{\operandTemporary{p}}{\temps{p}}{temporary that is connected to operand $p$}
  \tableSpace
  \modelVariable{\operationIssueCycle{o}}{\naturalNumbersZero}{issue cycle of operation $o$ relative
    to the beginning of its block}
  \tableSpace
  \modelVariable{\temporaryLiveStart{t}}{\naturalNumbersZero}{live start of temporary
    $t$}
  \tableSpace
  \modelVariable{\temporaryLiveEnd{t}}{\naturalNumbersZero}{live end of temporary $t$}
\end{tabular}


\section{Constraints}\label{sec:constraints}


\newcommand{\connectedOperandEquation}{
  \begin{equation}\textstyle\label{eq:connected-operand}
    \connectedOperand{p} \iff \operandTemporary{p} \neq \nullConnection{}
    \quad
    \forallIn{p}{\allOperands{}}
  \end{equation}
}

\newcommand{\liveStartEquation}{
\begin{equation}\textstyle\label{eq:live-start}
  \liveTemporary{t} \implies \temporaryLiveStart{t} =
  \operationIssueCycle{\operandOperation{\definer{t}}}
  \quad
  \forallIn{t}{\allTemporaries{}}
\end{equation}
}

\newcommand{\liveEndEquation}{
\begin{equation}\displaystyle\label{eq:live-end}
  \liveTemporary{t} \implies \temporaryLiveEnd{t} =
  \operator{max}\left(\;\;\;\maximum{\substack{p \in \users{t} \suchThat\\\operandTemporary{p} = t}}{\operationIssueCycle{\operandOperation{p}}}, \temporaryLiveStart{t} + \minimumLiveDuration{t}\right)
  \quad
  \forallIn{t}{\allTemporaries{}}
\end{equation}
}

\newcommand{\usersEquation}{
  \begin{equation}\textstyle\label{eq:users}
    \users{t} = \setBuilder{p \in \allOperands{}}{\use{p} \land t \in \temps{p}}
  \end{equation}
}

\newcommand{\connectedDefinerEquation}{
\begin{equation}\textstyle\label{eq:connected-definer}
  \liveTemporary{t} \iff \connectedOperand{\definer{t}}
  \quad
  \forallIn{t}{\allTemporaries{}}
\end{equation}
}

\newcommand{\definerEquation}{
  \begin{equation}\textstyle\label{eq:definer}
    \definer{t} = p \in \allOperands{} \suchThat \p{\neg \use{p} \land t \in \temps{p}}
  \end{equation}
}

\newcommand{\localOperandConnectionEquation}{
  \begin{equation}\textstyle\label{eq:local-operand-connection}
    \connectedOperand{p} \iff \activeOperation{\operandOperation{p}}
    \quad
    \forallIn{p}{\allOperands{}} \suchThat \neg \globalOperand{p}
  \end{equation}
}

\newcommand{\globalEquation}{
  \begin{equation}\textstyle\label{eq:global}
    \globalOperand{p} \iff \existsIn{q}{P} \suchThat \p{\adjacent{p}{q} \lor \adjacent{q}{p}}
    \quad
    \forallIn{p}{\allOperands{}}
  \end{equation}
}

\newcommand{\operandOperationEquation}{
  \begin{equation}\textstyle\label{eq:operand-operation}
     \operandOperation{p} = o \in \allOperations{} \suchThat p \in \operands{o}
  \end{equation}
}

\newcommand{\globalOperandConnectionEquation}{
  \begin{equation}\textstyle\label{eq:global-operand-connection}
    \begin{aligned}
      \connectedOperand{p} \iff
      \existsIn{q}{P} \suchThat
      \p{\adjacent{p}{q} \land \connectedOperand{q}}
      \quad
      \forallIn{p}{\allOperands{}} \suchThat \globalOperand{p}
    \end{aligned}
  \end{equation}
}

\newcommand{\activeInstructionEquation}{
  \begin{equation}\textstyle\label{eq:active-instruction}
    \activeOperation{o} \iff \operationInstruction{o} \neq \nullInstruction
    \quad
    \forallIn{o}{\allOperations{}}
  \end{equation}
}

\newcommand{\connectedUsersEquation}{
  \begin{equation}\textstyle\label{eq:connected-users}
    \liveTemporary{t} \iff \existsIn{p}{\users{t}} \suchThat
    \operandTemporary{p} = t
    \quad
    \forallIn{t}{\allTemporaries{}}
  \end{equation}
}

\newcommand{\preAssignmentEquation}{
\begin{equation}\textstyle\label{eq:pre-assignment}
  \temporaryRegister{\operandTemporary{p}} = \register{r}
  \quad
  \forallIn{p}{\allOperands{}} \suchThat \preAssigned{p\hspace{0.025cm}}{r}
\end{equation}
}

\newcommand{\preschedulingEquation}{
\begin{equation}\textstyle\label{eq:prescheduling}
  \activeOperation{o} \implies \operationIssueCycle{o} = \cycle{c}
  \quad
  \forallIn{o}{\allOperations{}} \suchThat \prescheduled{o}{c}
\end{equation}
and
\begin{equation}
  \activeOperation{o} \iff \cycle{c} < \operationIssueCycle{\operator{out}(\operationBlock{o})}
  \quad
  \forallIn{o}{\allOperations{}} \suchThat \prescheduled{o}{c}
\end{equation}
}

\newcommand{\bypassingEquation}{
\begin{equation}\textstyle\label{eq:bypassing}
  \bypassing{o}{\operationInstruction{o}}{p} \implies
  \operationIssueCycle{o} = \operationIssueCycle{\operandOperation{\definer{\operandTemporary{p}}}}
  \quad
  \forallIn{o}{\allOperations{}}, \,
  \forallIn{p}{\operands{o}}
\end{equation}
}

\newcommand{\disjointLiveRangesEquation}{
\begin{equation}\textstyle\label{eq:disjoint-live-ranges}
  \disjoint{
    \set{\tuple{
        \temporaryRegister{t},
        \temporaryRegister{t} + \width{t} \times \liveTemporary{t},
        \temporaryLiveStart{t},
        \temporaryLiveEnd{t}}
      \! \suchThat \!
      t \in \temporaries{b}}}
  \;\;
  \forallIn{b}{\allBlocks{}}
\end{equation}
}

\newcommand{\blockTemporariesEquation}{
  \begin{equation}\textstyle\label{eq:block-temporaries}
    \temporaries{b} = \setBuilder{t \in T}{\operationBlock{\operandOperation{\definer{t}}} = b}
  \end{equation}
}

\newcommand{\registerClassEquation}{
\begin{equation}\textstyle\label{eq:instruction-selection}
  \temporaryRegister{\operandTemporary{p}}
  \in
  \registerClass{o}{\operationInstruction{o}}{p}
  \quad
  \forallIn{o}{\allOperations{}}, \,
  \forallIn{p}{\operands{o}}
\end{equation}
}

\newcommand{\congruenceEquation}{
  \begin{equation}\textstyle\label{eq:congruence}
    \connectedOperand{p} \land
    \connectedOperand{q} \implies
    \temporaryRegister{\operandTemporary{p}} = \temporaryRegister{\operandTemporary{q}}
    \quad
    \forallIn{p, q}{\allOperands{}} \suchThat \adjacent{p}{q}
  \end{equation}
}

\newcommand{\alignmentEquation}{
\begin{equation}\textstyle\label{eq:alignment}
  \begin{aligned}
    \operationInstruction{\operandOperation{p}} = i \land
    \operationInstruction{\operandOperation{q}} = j \implies
    \temporaryRegister{\operandTemporary{p}} = \temporaryRegister{\operandTemporary{q}} + \alignmentDistance{p}{i}{q}{j} \\
    \forallIn{p, q}{\allOperands{}}, \,
    \forallIn{i, j}{I}
    \suchThat \alignedOperands{p}{i}{q}{j}
  \end{aligned}
\end{equation}
}

\newcommand{\packingEquation}{
\begin{equation}\textstyle\label{eq:packing}
  \begin{aligned}
    \connectedOperand{p} \land \connectedOperand{q}
    \implies
    &\temporaryRegister{\operandTemporary{q}} =
    \temporaryRegister{\operandTemporary{p}} +
    \begin{cases}
      \displaystyle{} \width{p} &
      \text{if }
      \temporaryRegister{\operandTemporary{p}}
      \eat \eat \eat \eat \mod \eat \p{\width{p} \times 2} = 0 \\
      \displaystyle{} - \width{p} &
      \text{otherwise}
    \end{cases}\\
    &\hspace{5cm}\forallIn{p, q}{\allOperands{}} \suchThat \packedOperands{p}{q}
  \end{aligned}
\end{equation}
}

\newcommand{\extensionalEquation}{
\begin{equation}\textstyle\label{eq:extensional}
  \extensional{\tuple{p,q},\extensionalTable{p}{q}}
  \quad
  \forallIn{p, q}{\allOperands{}} \suchThat \extensionalOperands{p}{q}
\end{equation}
}

\newcommand{\dataPrecedencesEquation}{
\begin{equation}\textstyle\label{eq:data-precedences}
  \begin{aligned}
    \operandTemporary{q} = t &\implies c_u \ge c_d + \latency{d}{\operationInstruction{d}}{p} + \slack{p} + \latency{u}{\operationInstruction{u}}{q} + \slack{q}\\
    & \forall t \in \allTemporaries{},\\
    & \forall p \in \set{\definer{t}}, \forall d \in \set{\operandOperation{p}}\\
    & \forall q \in \users{t},         \forall u \in \set{\operandOperation{q}}
  \end{aligned}
\end{equation}
}

\newcommand{\slackEquation}{
  \begin{equation}\textstyle\label{eq:slack}
    \slack{p} =
    \begin{cases}
      \displaystyle{} \operandLatencySlack{p} &
      \text{if }
      \globalOperand{p} \\
      \displaystyle{} 0 &
      \text{otherwise}
    \end{cases}
  \end{equation}
}

\newcommand{\fixedPrecedencesEquation}{
\begin{equation}\textstyle\label{eq:fixed-precedences}
  \activeOperation{d} \land \activeOperation{u} \implies \operationIssueCycle{u} \ge \operationIssueCycle{d} + \minimumDistance{d}{u}{\operationInstruction{d}} \quad
  \forall b \in B, \,
  \forall \sequence{d,u} \in \dependencyGraph{b}
\end{equation}
}

\newcommand{\activationEquation}{
\begin{equation}\textstyle\label{eq:activation}
  \exists{} \, o' \in \allOperations{} \suchThat \operationInstruction{o'} \in \activators{o} \implies \activeOperation{o} \quad
  \forall o \in \allOperations{}
\end{equation}
}

\newcommand{\slackBalancingEquation}{
  \begin{equation}\textstyle\label{eq:slack-balancing}
    \operandLatencySlack{p} + \operandLatencySlack{q} = 0
    \quad
    \forallIn{p, q}{\allOperands{}} \suchThat \adjacent{p}{q}
  \end{equation}
}

\newcommand{\processorResourcesEquation}{
  \begin{eqnarray}\textstyle\label{eq:processor-resources}
    \cumulative{
      \setBuilder
          {\tuple{
              \operationIssueCycle{o} +
              \offset{\operationInstruction{o}}{\processorResource{r}}, \eat
              \units{\operationInstruction{o}}{\processorResource{r}},
              \duration{\operationInstruction{o}}{\processorResource{r}}
              \eat}}
          {\eat o \in \operations{b}},
          \capacity{\processorResource{r}}}\notag{}\hspace{-0.5cm}\\
    \forallIn{b}{\allBlocks{}}, \forallIn{\processorResource{r}}{R}
  \end{eqnarray}
}

\newcommand{\blockOperationsEquation}{
  \begin{equation}\textstyle\label{eq:block-operations}
    \operations{b} = \setBuilder{o \in O}{\operationBlock{o} = b}
  \end{equation}
}

\newcommand{\genericObjectiveEquation}{
  \begin{equation*}\label{eq:generic-objective}
    \summation{b \in B}{\blockWeight{b} \times \blockCost{b}}
  \end{equation*}
}

\newcommand{\exampleLabel}{\text{\emph{example:}} \quad}

\newcommand{\connectedOperandExample}{
  \begin{equation*}\textstyle
    \exampleLabel{}
    \connectedOperand{\operand{5}} \iff \operandTemporary{\operand{5}} \neq \nullConnection{}
  \end{equation*}
}

\newcommand{\connectedUsersExample}{
  \begin{equation*}\textstyle
    \exampleLabel{}
    \liveTemporary{\temporary{4}} \iff \existsIn{p}{\set{\operand{13}, \operand{22}}} \suchThat
    \operandTemporary{p} = \temporary{4}
  \end{equation*}
}

\newcommand{\connectedDefinerExample}{
  \begin{equation*}\textstyle
    \exampleLabel{}
    \liveTemporary{\temporary{19}} \iff \connectedOperand{\operand{37}}
  \end{equation*}
}


\subsection{Register Allocation}

\constraintComment{connected operands}{operands cannot be connected to null temporaries}
\connectedOperandEquation
\connectedOperandExample
\constraintComment{connected users}{a temporary is live iff it is connected to a user}
\connectedUsersEquation
where
\usersEquation
\connectedUsersExample
\constraintComment{connected definers}{a temporary is live iff it is connected to its definer}
\connectedDefinerEquation
where
\definerEquation
\connectedDefinerExample
\constraintComment{local operand connections}{local operands are connected iff their operations are active}
\localOperandConnectionEquation
where
\globalEquation
and
\operandOperationEquation
\localOperandConnectionExample
\constraintComment{global operand connections}{global operands are connected iff any of their successors is connected}
\globalOperandConnectionEquation
\globalOperandConnectionExample
\constraintComment{active instructions}{active operations are implemented by non-null instructions}
\activeInstructionEquation
\activeInstructionExample
\constraintComment{register class}{the instruction that implements an
  operation determines the register class to which its operands are allocated}
\registerClassEquation
\registerClassExample
\constraintComment{disjoint live ranges}{temporaries whose live ranges overlap are assigned to different register atoms}
\disjointLiveRangesEquation
where
\blockTemporariesEquation
\disjointLiveRangesExample
\constraintComment{preassignment}{certain operands are preassigned to registers}
\preAssignmentEquation
\preAssignmentExample
\constraintComment{congruence}{connected adjacent operands are assigned to the same register}
\congruenceEquation
\congruenceExample
\constraintComment{alignment}{aligned operands are assigned to registers at a
  given relative distance}
\alignmentEquation
\constraintComment{packing}{packed operands are assigned to contiguous, complementary registers}
\packingEquation
\constraintComment{extensional}{the registers assigned to some pairs of operands are related extensionally}
\extensionalEquation

\subsection{Instruction Scheduling}

\constraintComment{live start}{the live range of a temporary starts at the issue cycle of its definer}
\liveStartEquation
\constraintComment{live end}{the live range of a temporary ends with the last issue cycle of its users}
\liveEndEquation
\constraintComment{data precedences}{an operation that uses a temporary must be preceded by its definer}
\dataPrecedencesEquation
where
\slackEquation
\constraintComment{processor resources}{the capacity of each processor resource cannot be exceeded at any issue cycle}
\processorResourcesEquation
where
\blockOperationsEquation
\constraintComment{fixed precedences}{control and read-write dependencies yield fixed precedences among operations}
\fixedPrecedencesEquation
\constraintComment{activation}{an operation is active if any of its activator instructions is selected}
\activationEquation
\constraintComment{slack balancing}{the slack of adjacent operands is balanced}
\slackBalancingEquation
\constraintComment{prescheduling}{certain operations are prescheduled before the last operation issue}
\preschedulingEquation
\constraintComment{bypassing}{the operation of a bypassing operand is scheduled in parallel with its definer}
\bypassingEquation
\constraintComment{adhoc}{ad hoc processor constraints are satisfied}
\adhocEquation
where
\satisfyEquation

\section{Objective}\label{sec:objective}

The objective is to minimize the sum of the weighted costs of each block
according to the first objective (multi-objective optimization is not yet
supported):
%
\genericObjectiveEquation
%
where $\blockWeight{b}$ gives the weight of block $b$:
%
\blockWeigthEquation
%
and $\blockCost{b}$ gives the estimated cost of block $b$:
%
\blockCostEquation

To optimize for speed, $\optimizeDynamic{0}$ is set to $\code{true}$ and
$\optimizeResource{0}$ is set to the special resource $\code{cycles}$.
%
To optimize for code size, $\optimizeDynamic{0}$ is set to $\code{false}$
and $\optimizeResource{0}$ is set to the processor resource $\code{bits}$
representing the bits with which instructions are encoded.

\chapter{Target Description}
\label{sec:target-description}

\appendix

\chapter{Further Reading}
\label{sec:further-reading}

\bibliographystyle{plain}
\bibliography{manual}

\end{document}
