\documentclass{article}
\usepackage{graphicx}
\usepackage{abstract}
\usepackage[latin1]{inputenc}
\usepackage[english]{babel}
\usepackage[breaklinks]{hyperref}
\usepackage{url}
\usepackage{etoolbox}
\usepackage{amsmath}
\usepackage{unison}
\usepackage[adobe-utopia]{mathdesign}

\setcounter{secnumdepth}{2}

\title{The Unison Manual}

\author{Roberto Casta�eda Lozano}

\date{}

\renewcommand{\abstractname}{}
\renewcommand{\absnamepos}{empty}

\begin{document}

\maketitle

\setcounter{tocdepth}{2}
\tableofcontents

\section{Introduction}

\section{Getting Started}

\section{LLVM Integration}

\section{Architecture}

\section{Unison IR}

\section{Constraint Model}\label{sec:constraint-model}

\subsection{Parameters}\label{sec:parameters}

\boolfalse{showJson}

\subsubsection{Program}

\begin{longtable}{ l p{12cm} }

  \parameter{B, O, P, T}{sets of blocks, operations, operands and temporaries}
  \json{B}{[0, 1, 2]}
  \json{O}{[0, 1, 2, \dots, 33, 34]}
  \json{P}{[0, 1, 2, \dots, 67, 68]}
  \json{T}{[0, 1, 2, \dots, 31, 32]}

  \tableSpace

  \parameter{\operationBlock{o}}{block to which operation $o$ belongs}
  \json{block}{[0, 0, \dots, 1, 1, 1, 2, 2, 2, 2, 2]}

  \tableSpace

  \parameter{\operands{o}}{set of operands of operation $o$}
  \json{operands}{[[0, 1], [2, 3], [4], \dots, [65, 66], [67], [68]]}

  \tableSpace

  \parameter{\temps{p}}{set of temporaries that can be connected to operand $p$}
  \json{temps}{[[0], [1], [-1, 0], \dots, [30], [29, 31, 32]]}

  \tableSpace

  \parameter{\use{p}}{whether $p$ is a use operand}
  \json{use}{[false, false, true, \dots, true, true]}

  \tableSpace

  \parameter{\adjacent{p}{q}}{whether operands $p$ and $q$ are adjacent}
  \json{adjacent}{[[20, 24], [21, 25], \dots, [60, 61]]}

  \tableSpace

  \parameter
  {\preAssigned{p\hspace{0.025cm}}{r}}
  {whether operand $p$ is preassigned to register $\register{r}$}
  \json{preassign}{[[0, 0], [1, 31], [68, 0]]}

  \tableSpace

  \parameter{\width{t}}{number of register atoms that temporary $t$ occupies}
  \json{width}{[1, 1, 1, \dots, 1, 1]}

  \tableSpace

  \parameter{\frequency{b}}{estimated execution frequency of block $b$}
  \json{freq}{[4, 85, 4]}

  \tableSpace

  % Below parameters are not in LCTES2014 paper

  \parameter{\alignedOperands{p}{i}{q}{j}}{whether operands $p$ and $q$ are
    aligned when implemented by instructions $i$ and $j$}
  \json{aligned}{[[69, 17, 71, 17], [70, 24, 71, 24]]}
  \jsonComment{note: the example JSON arrays are extracted from a different
    program (\code{factorial} does not contain alignment constraints)}

  \tableSpace

  \parameter{\alignmentDistance{p}{i}{q}{j}}{alignment distance of operands $p$
    and $q$ when implemented by instructions $i$ and $j$}
  \json{adist}{[0, 1, 1]}
  \jsonComment{note: this parameter is encoded with the same structure as
    \code{aligned}: each aligned operand tuple and its corresponding alignment
    distance are found in the same positions of their respective JSON arrays
    (example: \code{aligned[1]~=~[70, 24, 71, 24]}, \code{adist[1]~=~1}.)}

  \tableSpace

  \parameter{\packedOperands{p}{q}}{whether operands $p$ and $q$ are
    packed}
  \json{packed}{[[13, 14], [34, 35], [54, 55]]}
  \jsonComment{note: the example JSON arrays are extracted from a different
    program (\code{factorial} does not contain packing constraints)}

  \tableSpace

  \parameter{\minimumLiveDuration{t}}{minimum live duration of temporary $t$ if
    it is live}
  \json{minlive}{[1, 1, 1, \dots, 1, 1]}

  \tableSpace

  \parameter
  {\dependencyGraph{b}}
  {fixed dependency graph of the operations of block $b$}
  \json{dep}{[[[0, 1], \dots, [1, 8], \dots, [10, 11]], [\dots], [\dots]]}

  \tableSpace

  \parameter{\activators{o}}{set of instructions that activate operation $o$}
  \json{activators}{[[], [10, 17, 13, 19], [10, 17, 13, 19], \dots, []]}
  \jsonComment{note: the example JSON array is extracted from a different
    program (Hexagon programs do not yet yield activation constraints)}

  \tableSpace

  \parameter
  {\participative{o}{c}}
    {whether operation $o$ is \emph{participative} with cycle $\cycle{c}$}
  \json{part}{[[2, 3], [1, 1], [60, 24]]}

\end{longtable}


\subsubsection{Processor}

\begin{longtable}{ l p{12cm} }

  \parameter{I, R}{sets of instructions and resources}
  \json{I}{[0, 1, 2, 3, \dots, 16]}
  \json{R}{[0, 1, 2, 3, \dots, 8]}

  \tableSpace

  \parameter{\minimumDistance{o_1}{o_2}{\instruction{i}}}{min.~issue distance of
    ops.~$o_1$ and $o_2$ when $o_1$ is implemented by $\instruction{i}$}
  \json{dist}{[[[1], \dots, [0, 0, 0], \dots, [1]], [\dots], [\dots]]}
  \jsonComment{note: this parameter is encoded with the same structure as
    \code{dep}: each dependency and its corresponding distance array are found
    in the same positions of their respective JSON arrays (example:
    \code{dep[0][2]~=~[0, 3]}, \code{dist[0][2] = [1]}).}

  \tableSpace

  \parameter{\registerClass{o}{\instruction{i}}{p}}{register class in which
    operation $o$ implemented by $\instruction{i}$ accesses $p$}
  \json{class}{[[0, 0]], [[0, 0], [1, 1], [1, 9]], \dots, [[0]]}

  \tableSpace

  \parameter{\atoms{rc}}{atoms of register class $rc$}
  \json{atoms}{[[0, 1, 2, \dots, 76], \dots, [37, 39, 41, \dots, 75]]}

  \tableSpace

  \parameter{\instructions{o}}{set of instructions that can implement operation
    $o$}
  \json{instructions}{[[2], [0, 3, 4], [5], \dots, [16], [2]]}

  \tableSpace

  \parameter{\latency{o}{\instruction{i}}{p}}{latency of $p$ when its operation $o$ is implemented by $\instruction{i}$}
  \json{lat}{[[[1, 1]], [[0, 0], [0, 1], [0, 1]], \dots, [[0]]]}

  \tableSpace

  \parameter{\bypassing{o}{\instruction{i}}{p}}{whether $p$ is bypassing when its operation $o$ is implemented by $\instruction{i}$}
  \json{bypass}{[[[false, false]], [[false, false], \dots, [[false]]]}

  \tableSpace

  \parameter{\capacity{\code{r}}}{capacity of processor resource $r$}
  \json{cap}{[4, 4, 2, 1, 2, 1, 1, 2, 1]}

  \tableSpace

  \parameter{\units{\instruction{i}}{\code{r}}}
  {consumption of processor resource $r$ by instruction $\instruction{i}$}
  \json{con}{[[0, 0, \dots, 0, 0], \dots, [1, 1, 0, 0, 1, 1, 0, 0, 0]]}

  \tableSpace

  \parameter{\duration{\instruction{i}}{\code{r}}}{duration of usage of processor
    resource $r$ by instruction $\instruction{i}$}
  \json{dur}{[[0, 0, \dots, 0, 0], \dots, [1, 1, 0, 0, 1, 1, 0, 0, 0]]}

  \tableSpace

  \parameter{\offset{\instruction{i}}{\code{r}}}{offset of usage of processor
    resource $r$ by instruction $\instruction{i}$}
  \json{off}{[[0, 0, \dots, 0, 0], \dots, [0, 0, 0, 0, 0, 0, 0, 0, 0]]}

  % Below parameters are not in LCTES2014 paper

  \parameter{\alignedOperands{p}{i}{q}{j}}{whether operands $p$ and $q$ are
    aligned when implemented by instructions $i$ and $j$}
  \json{aligned}{[[69, 17, 71, 17], [70, 24, 71, 24]]}
  \jsonComment{note: the example JSON arrays are extracted from a different
    program (Hexagon programs do not yield alignment constraints)}

  \tableSpace

  \parameter{\alignmentDistance{p}{i}{q}{j}}{alignment distance of operands $p$
    and $q$ when implemented by instructions $i$ and $j$}
  \json{adist}{[0, 1, 1]}
  \jsonComment{note: this parameter is encoded with the same structure as
    \code{aligned}: each aligned operand tuple and its corresponding alignment
    distance are found in the same positions of their respective JSON arrays
    (example: \code{aligned[1]~=~[70, 24, 71, 24]}, \code{adist[1]~=~1}.)}

  \tableSpace

  \parameter{\packedOperands{p}{q}}{whether operands $p$ and $q$ are
    packed}
  \json{packed}{[[13, 14], [34, 35], [54, 55]]}
  \jsonComment{note: the example JSON arrays are extracted from a different
    program (Hexagon programs do not yield packing constraints)}

  \tableSpace

  \parameter{\extensionalOperands{p}{q}}{whether operands $p$ and $q$ are
    related extensionally}
  \json{exrelated}{[[4, 5]]}
  \jsonComment{note: the example JSON arrays are extracted from a different
    program (Hexagon programs do not yield extensional constraints)}

  \tableSpace

  \parameter{\extensionalTable{p}{q}}{table of register assignments for
    operands $p$ and $q$}
  \json{table}{[[0, 1],[2, 3],[4, 5],[6, 7]]}
  \jsonComment{note: the example JSON arrays are extracted from a different
    program (Hexagon programs do not yield extensional constraints)}

  \tableSpace

  \parameter{\activators{o}}{set of instructions that activate operation $o$}
  \json{activators}{[[], [10, 17, 13, 19], [10, 17, 13, 19], \dots, []]}
  \jsonComment{note: the example JSON array is extracted from a different
    program (Hexagon programs do not yet yield activation constraints)}

  \tableSpace

  \parameter{\callerSavedAtoms{}}{caller-saved atoms}
  \json{callersaved}{[0, 1, 2, 3, 4, 5, 6, 7, 8, 9, 10, 11, 12, 13, 14, 15, 28, 32, 33, 34, 35]}

  \tableSpace

  \parameter{\allAdhocConstraints{}}{set of ad hoc processor constraints}
  \json{E}{[[2, [1, [5, 13], [5, 14]], [5, 15]], \dots]}
  \jsonComment{note: constraints are encoded as trees of expression tuples,
    where the first element of each expression tuple encodes its type as
    follows:}
  \jsonComment{%
    \mbox{%
      \begin{tabular}{|l l|}
        \hline
        \begin{tabular}{ l l }
          \code{0} & \code{or}\\
          \code{1} & \code{and}\\
          \code{2} & \code{xor}\\
          \code{3} & \code{implies}\\
          \code{4} & \code{not}\\
          \code{5} & \code{active}\\
          \code{6} & \code{connects}\\
        \end{tabular}
        &
        \begin{tabular}{ l l }
          \code{7} & \code{implements}\\
          \code{8} & \code{distance}\\
          \code{9} & \code{share}\\
          \code{10} & \code{operand-overlap}\\
          \code{11} & \code{temporary-overlap}\\
          \code{12} & \code{caller-saved}\\
          \code{13} & \code{allocated}\\
        \end{tabular}\\
        \hline
      \end{tabular}
    }
  }
  \jsonComment{note: the example JSON array is extracted from a different
    program (Hexagon programs do not yet yield ad hoc constraints)}

\end{longtable}


\subsubsection{Objective}

\begin{tabular}{ l p{9.7cm} }

  \parameter{\optimizeDynamic{}}{whether to use block frequencies as weight}
  \json{optimize\_dynamic}{true}

  \tableSpace

  \parameter{\optimizeResource{}}{resource whose consumption is to be optimized}
  \json{optimize\_resource}{-1}
  \jsonComment{Note: the estimated number of cycles is encoded as resource
    \code{-1}, otherwise the usual resource numbers in \code{R} are used}

  \tableSpace

  \parameter{\maxCost{}}{upper bound of the objective}
  \json{maxf}{274}

\end{tabular}


\subsection{Variables}

\begin{tabular}{ l p{12cm} }
  \modelVariable{\activeOperation{o}}{\set{0, 1}}{whether operation $o$ is active}
  \tableSpace
  \modelVariable{\operationInstruction{o}}{\instructions{o}}{instruction that implements operation $o$}
  \tableSpace
  \modelVariable{\liveTemporary{t}}{\set{0, 1}}{whether temporary $t$ is live}
  \tableSpace
  \modelVariable{\temporaryRegister{t}}{\naturalNumbersZero}{register to which temporary $t$ is assigned}
  \tableSpace
  \modelVariable{\connectedOperand{p}}{\set{0, 1}}{whether operand $p$ is connected}
  \tableSpace
  \modelVariable{\operandTemporary{p}}{\temps{p}}{temporary that is connected to operand $p$}
  \tableSpace
  \modelVariable{\operationIssueCycle{o}}{\naturalNumbersZero}{issue cycle of operation $o$ relative
    to the beginning of its block}
  \tableSpace
  \modelVariable{\temporaryLiveStart{t}}{\naturalNumbersZero}{live start of temporary
    $t$}
  \tableSpace
  \modelVariable{\temporaryLiveEnd{t}}{\naturalNumbersZero}{live end of temporary $t$}
\end{tabular}


\subsection{Constraints}


\newcommand{\connectedOperandEquation}{
  \begin{equation}\textstyle\label{eq:connected-operand}
    \connectedOperand{p} \iff \operandTemporary{p} \neq \nullConnection{}
    \quad
    \forallIn{p}{\allOperands{}}
  \end{equation}
}

\newcommand{\liveStartEquation}{
\begin{equation}\textstyle\label{eq:live-start}
  \liveTemporary{t} \implies \temporaryLiveStart{t} =
  \operationIssueCycle{\operandOperation{\definer{t}}}
  \quad
  \forallIn{t}{\allTemporaries{}}
\end{equation}
}

\newcommand{\liveEndEquation}{
\begin{equation}\displaystyle\label{eq:live-end}
  \liveTemporary{t} \implies \temporaryLiveEnd{t} =
  \operator{max}\left(\;\;\;\maximum{\substack{p \in \users{t} \suchThat\\\operandTemporary{p} = t}}{\operationIssueCycle{\operandOperation{p}}}, \temporaryLiveStart{t} + \minimumLiveDuration{t}\right)
  \quad
  \forallIn{t}{\allTemporaries{}}
\end{equation}
}

\newcommand{\usersEquation}{
  \begin{equation}\textstyle\label{eq:users}
    \users{t} = \setBuilder{p \in \allOperands{}}{\use{p} \land t \in \temps{p}}
  \end{equation}
}

\newcommand{\connectedDefinerEquation}{
\begin{equation}\textstyle\label{eq:connected-definer}
  \liveTemporary{t} \iff \connectedOperand{\definer{t}}
  \quad
  \forallIn{t}{\allTemporaries{}}
\end{equation}
}

\newcommand{\definerEquation}{
  \begin{equation}\textstyle\label{eq:definer}
    \definer{t} = p \in \allOperands{} \suchThat \p{\neg \use{p} \land t \in \temps{p}}
  \end{equation}
}

\newcommand{\localOperandConnectionEquation}{
  \begin{equation}\textstyle\label{eq:local-operand-connection}
    \connectedOperand{p} \iff \activeOperation{\operandOperation{p}}
    \quad
    \forallIn{p}{\allOperands{}} \suchThat \neg \globalOperand{p}
  \end{equation}
}

\newcommand{\globalEquation}{
  \begin{equation}\textstyle\label{eq:global}
    \globalOperand{p} \iff \existsIn{q}{P} \suchThat \p{\adjacent{p}{q} \lor \adjacent{q}{p}}
    \quad
    \forallIn{p}{\allOperands{}}
  \end{equation}
}

\newcommand{\operandOperationEquation}{
  \begin{equation}\textstyle\label{eq:operand-operation}
     \operandOperation{p} = o \in \allOperations{} \suchThat p \in \operands{o}
  \end{equation}
}

\newcommand{\globalOperandConnectionEquation}{
  \begin{equation}\textstyle\label{eq:global-operand-connection}
    \begin{aligned}
      \connectedOperand{p} \iff
      \existsIn{q}{P} \suchThat
      \p{\adjacent{p}{q} \land \connectedOperand{q}}
      \quad
      \forallIn{p}{\allOperands{}} \suchThat \globalOperand{p}
    \end{aligned}
  \end{equation}
}

\newcommand{\activeInstructionEquation}{
  \begin{equation}\textstyle\label{eq:active-instruction}
    \activeOperation{o} \iff \operationInstruction{o} \neq \nullInstruction
    \quad
    \forallIn{o}{\allOperations{}}
  \end{equation}
}

\newcommand{\connectedUsersEquation}{
  \begin{equation}\textstyle\label{eq:connected-users}
    \liveTemporary{t} \iff \existsIn{p}{\users{t}} \suchThat
    \operandTemporary{p} = t
    \quad
    \forallIn{t}{\allTemporaries{}}
  \end{equation}
}

\newcommand{\preAssignmentEquation}{
\begin{equation}\textstyle\label{eq:pre-assignment}
  \temporaryRegister{\operandTemporary{p}} = \register{r}
  \quad
  \forallIn{p}{\allOperands{}} \suchThat \preAssigned{p\hspace{0.025cm}}{r}
\end{equation}
}

\newcommand{\preschedulingEquation}{
\begin{equation}\textstyle\label{eq:prescheduling}
  \activeOperation{o} \implies \operationIssueCycle{o} = \cycle{c}
  \quad
  \forallIn{o}{\allOperations{}} \suchThat \prescheduled{o}{c}
\end{equation}
and
\begin{equation}
  \activeOperation{o} \iff \cycle{c} < \operationIssueCycle{\operator{out}(\operationBlock{o})}
  \quad
  \forallIn{o}{\allOperations{}} \suchThat \prescheduled{o}{c}
\end{equation}
}

\newcommand{\bypassingEquation}{
\begin{equation}\textstyle\label{eq:bypassing}
  \bypassing{o}{\operationInstruction{o}}{p} \implies
  \operationIssueCycle{o} = \operationIssueCycle{\operandOperation{\definer{\operandTemporary{p}}}}
  \quad
  \forallIn{o}{\allOperations{}}, \,
  \forallIn{p}{\operands{o}}
\end{equation}
}

\newcommand{\disjointLiveRangesEquation}{
\begin{equation}\textstyle\label{eq:disjoint-live-ranges}
  \disjoint{
    \set{\tuple{
        \temporaryRegister{t},
        \temporaryRegister{t} + \width{t} \times \liveTemporary{t},
        \temporaryLiveStart{t},
        \temporaryLiveEnd{t}}
      \! \suchThat \!
      t \in \temporaries{b}}}
  \;\;
  \forallIn{b}{\allBlocks{}}
\end{equation}
}

\newcommand{\blockTemporariesEquation}{
  \begin{equation}\textstyle\label{eq:block-temporaries}
    \temporaries{b} = \setBuilder{t \in T}{\operationBlock{\operandOperation{\definer{t}}} = b}
  \end{equation}
}

\newcommand{\registerClassEquation}{
\begin{equation}\textstyle\label{eq:instruction-selection}
  \temporaryRegister{\operandTemporary{p}}
  \in
  \registerClass{o}{\operationInstruction{o}}{p}
  \quad
  \forallIn{o}{\allOperations{}}, \,
  \forallIn{p}{\operands{o}}
\end{equation}
}

\newcommand{\congruenceEquation}{
  \begin{equation}\textstyle\label{eq:congruence}
    \connectedOperand{p} \land
    \connectedOperand{q} \implies
    \temporaryRegister{\operandTemporary{p}} = \temporaryRegister{\operandTemporary{q}}
    \quad
    \forallIn{p, q}{\allOperands{}} \suchThat \adjacent{p}{q}
  \end{equation}
}

\newcommand{\alignmentEquation}{
\begin{equation}\textstyle\label{eq:alignment}
  \begin{aligned}
    \operationInstruction{\operandOperation{p}} = i \land
    \operationInstruction{\operandOperation{q}} = j \implies
    \temporaryRegister{\operandTemporary{p}} = \temporaryRegister{\operandTemporary{q}} + \alignmentDistance{p}{i}{q}{j} \\
    \forallIn{p, q}{\allOperands{}}, \,
    \forallIn{i, j}{I}
    \suchThat \alignedOperands{p}{i}{q}{j}
  \end{aligned}
\end{equation}
}

\newcommand{\packingEquation}{
\begin{equation}\textstyle\label{eq:packing}
  \begin{aligned}
    \connectedOperand{p} \land \connectedOperand{q}
    \implies
    &\temporaryRegister{\operandTemporary{q}} =
    \temporaryRegister{\operandTemporary{p}} +
    \begin{cases}
      \displaystyle{} \width{p} &
      \text{if }
      \temporaryRegister{\operandTemporary{p}}
      \eat \eat \eat \eat \mod \eat \p{\width{p} \times 2} = 0 \\
      \displaystyle{} - \width{p} &
      \text{otherwise}
    \end{cases}\\
    &\hspace{5cm}\forallIn{p, q}{\allOperands{}} \suchThat \packedOperands{p}{q}
  \end{aligned}
\end{equation}
}

\newcommand{\extensionalEquation}{
\begin{equation}\textstyle\label{eq:extensional}
  \extensional{\tuple{p,q},\extensionalTable{p}{q}}
  \quad
  \forallIn{p, q}{\allOperands{}} \suchThat \extensionalOperands{p}{q}
\end{equation}
}

\newcommand{\dataPrecedencesEquation}{
\begin{equation}\textstyle\label{eq:data-precedences}
  \begin{aligned}
    \operandTemporary{q} = t &\implies c_u \ge c_d + \latency{d}{\operationInstruction{d}}{p} + \slack{p} + \latency{u}{\operationInstruction{u}}{q} + \slack{q}\\
    & \forall t \in \allTemporaries{},\\
    & \forall p \in \set{\definer{t}}, \forall d \in \set{\operandOperation{p}}\\
    & \forall q \in \users{t},         \forall u \in \set{\operandOperation{q}}
  \end{aligned}
\end{equation}
}

\newcommand{\slackEquation}{
  \begin{equation}\textstyle\label{eq:slack}
    \slack{p} =
    \begin{cases}
      \displaystyle{} \operandLatencySlack{p} &
      \text{if }
      \globalOperand{p} \\
      \displaystyle{} 0 &
      \text{otherwise}
    \end{cases}
  \end{equation}
}

\newcommand{\fixedPrecedencesEquation}{
\begin{equation}\textstyle\label{eq:fixed-precedences}
  \activeOperation{d} \land \activeOperation{u} \implies \operationIssueCycle{u} \ge \operationIssueCycle{d} + \minimumDistance{d}{u}{\operationInstruction{d}} \quad
  \forall b \in B, \,
  \forall \sequence{d,u} \in \dependencyGraph{b}
\end{equation}
}

\newcommand{\activationEquation}{
\begin{equation}\textstyle\label{eq:activation}
  \exists{} \, o' \in \allOperations{} \suchThat \operationInstruction{o'} \in \activators{o} \implies \activeOperation{o} \quad
  \forall o \in \allOperations{}
\end{equation}
}

\newcommand{\slackBalancingEquation}{
  \begin{equation}\textstyle\label{eq:slack-balancing}
    \operandLatencySlack{p} + \operandLatencySlack{q} = 0
    \quad
    \forallIn{p, q}{\allOperands{}} \suchThat \adjacent{p}{q}
  \end{equation}
}

\newcommand{\processorResourcesEquation}{
  \begin{eqnarray}\textstyle\label{eq:processor-resources}
    \cumulative{
      \setBuilder
          {\tuple{
              \operationIssueCycle{o} +
              \offset{\operationInstruction{o}}{\processorResource{r}}, \eat
              \units{\operationInstruction{o}}{\processorResource{r}},
              \duration{\operationInstruction{o}}{\processorResource{r}}
              \eat}}
          {\eat o \in \operations{b}},
          \capacity{\processorResource{r}}}\notag{}\hspace{-0.5cm}\\
    \forallIn{b}{\allBlocks{}}, \forallIn{\processorResource{r}}{R}
  \end{eqnarray}
}

\newcommand{\blockOperationsEquation}{
  \begin{equation}\textstyle\label{eq:block-operations}
    \operations{b} = \setBuilder{o \in O}{\operationBlock{o} = b}
  \end{equation}
}

\newcommand{\genericObjectiveEquation}{
  \begin{equation*}\label{eq:generic-objective}
    \summation{b \in B}{\blockWeight{b} \times \blockCost{b}}
  \end{equation*}
}


\subsubsection{Register allocation}

\constraintComment{connected operands}{operands cannot be connected to null temporaries}
\connectedOperandEquation
\constraintComment{connected users}{a temporary is live iff it is connected to a user}
\connectedUsersEquation
where
\usersEquation
\constraintComment{connected definers}{a temporary is live iff it is connected to its definer}
\connectedDefinerEquation
where
\definerEquation
\constraintComment{local operand connections}{local operands are connected iff their operations are active}
\localOperandConnectionEquation
where
\globalEquation
and
\operandOperationEquation
\constraintComment{global operand connections}{global operands are connected iff any of their successors is connected}
\globalOperandConnectionEquation
\constraintComment{active instructions}{active operations are implemented by non-null instructions}
\activeInstructionEquation
\constraintComment{register class}{The instruction that implements an
  operation determines the register class to which its operands are allocated}
\registerClassEquation
\constraintComment{disjoint live ranges}{temporaries whose live ranges overlap are assigned to different register atoms}
\disjointLiveRangesEquation
where
\blockTemporariesEquation
\constraintComment{preassignment}{certain operands are preassigned to registers}
\preAssignmentEquation
\constraintComment{congruence}{connected adjacent operands are assigned to the same register}
\congruenceEquation
\constraintComment{alignment}{aligned operands are assigned to registers at a
  given relative distance}
\alignmentEquation
\constraintComment{packing}{\emph{bound} operands are packed together
  with \emph{free} operands assigned to pack register classes}
\packingEquation


\subsubsection{Instruction Scheduling}

\constraintComment{live start}{the live range of a temporary starts at the issue cycle of its definer}
\liveStartEquation
\constraintComment{live end}{the live range of a temporary ends with the last issue cycle of its users}
\liveEndEquation
\constraintComment{data precedences}{an operation that uses a temporary must be preceded by its definer}
\dataPrecedencesEquation
\constraintComment{processor resources}{the capacity of each processor resource cannot be exceeded at any issue cycle}
\processorResourcesEquation
where
\blockOperationsEquation
\constraintComment{fixed precedences}{control and read-write dependencies yield fixed precedences among operations}
\fixedPrecedencesEquation
\constraintComment{activation}{an operation is active if any of its activator instructions is selected}
\activationEquation

\subsection{Objective}

The objective is to minimize the sum of the weigthed costs of each block:
%
\genericObjectiveEquation
%
where $\blockWeight{b}$ and $\blockCost{b}$ give the weight and estimated cost
of block $b$.
%
To optimize for speed, $\blockWeight{b}$ is set to $\blockFrequency{b}$
(corresponding to $\optimizeDynamic{} = \code{true}$), and $\blockCost{b}$ is defined
as $\max_{o \in \operations{b} \suchThat
  \activeOperation{o}}{\operationIssueCycle{o}}$ (or simply as
$\operationIssueCycle{\operatorname{out(b)}}$ where $\operatorname{out(b)}$ is
the out-delimiter of block $b$).
%
To optimize for code size, $\blockWeight{b}$ is disregarded (corresponding to
$\optimizeDynamic{} = \code{false}$) and $\blockCost{b}$ is defined as $\sum_{o
  \in \operations{b}}{\units{\operationInstruction{o}}{\bits{}}}$, where the
processor resource $\bits{}$ represents the bits with which instructions are
encoded (the particular resource to be optimized for is determined by the
parameter $\optimizeResource{}$).

\section{Target Description}

\end{document}
