\begin{longtable}{ l p{12cm} }

  \parameter{I, R}{sets of instructions and resources}
  \json{I}{[0, 1, 2, 3, \dots, 16]}
  \json{R}{[0, 1, 2, 3, \dots, 8]}

  \tableSpace

  \parameter{\minimumDistance{o_1}{o_2}{\instruction{i}}}{min.~issue distance of
    ops.~$o_1$ and $o_2$ when $o_1$ is implemented by $\instruction{i}$}
  \json{dist}{[[[1], \dots, [0, 0, 0], \dots, [1]], [\dots], [\dots]]}
  \jsonComment{note: this parameter is encoded with the same structure as
    \code{dep}: each dependency and its corresponding distance array are found
    in the same positions of their respective JSON arrays (example:
    \code{dep[0][2]~=~[0, 3]}, \code{dist[0][2] = [1]}).}

  \tableSpace

  \parameter{\registerClass{o}{\instruction{i}}{p}}{register class in which
    operation $o$ implemented by $\instruction{i}$ accesses $p$}
  \json{class}{[[0, 0]], [[0, 0], [1, 1], [1, 9]], \dots, [[0]]}

  \tableSpace

  \parameter{\atoms{rc}}{atoms of register class $rc$}
  \json{atoms}{[[0, 1, 2, \dots, 76], \dots, [37, 39, 41, \dots, 75]]}

  \tableSpace

  \parameter{\instructions{o}}{set of instructions that can implement operation
    $o$}
  \json{instructions}{[[2], [0, 3, 4], [5], \dots, [16], [2]]}

  \tableSpace

  \parameter{\latency{o}{\instruction{i}}{p}}{latency of $p$ when its operation $o$ is implemented by $\instruction{i}$}
  \json{lat}{[[[1, 1]], [[0, 0], [0, 1], [0, 1]], \dots, [[0]]]}

  \tableSpace

  \parameter{\bypassing{o}{\instruction{i}}{p}}{whether $p$ is bypassing when its operation $o$ is implemented by $\instruction{i}$}
  \json{bypass}{[[[false, false]], [[false, false], \dots, [[false]]]}

  \tableSpace

  \parameter{\capacity{\code{r}}}{capacity of processor resource $r$}
  \json{cap}{[4, 4, 2, 1, 2, 1, 1, 2, 1]}

  \tableSpace

  \parameter{\units{\instruction{i}}{\code{r}}}
  {consumption of processor resource $r$ by instruction $\instruction{i}$}
  \json{con}{[[0, 0, \dots, 0, 0], \dots, [1, 1, 0, 0, 1, 1, 0, 0, 0]]}

  \tableSpace

  \parameter{\duration{\instruction{i}}{\code{r}}}{duration of usage of processor
    resource $r$ by instruction $\instruction{i}$}
  \json{dur}{[[0, 0, \dots, 0, 0], \dots, [1, 1, 0, 0, 1, 1, 0, 0, 0]]}

  \tableSpace

  \parameter{\offset{\instruction{i}}{\code{r}}}{offset of usage of processor
    resource $r$ by instruction $\instruction{i}$}
  \json{off}{[[0, 0, \dots, 0, 0], \dots, [0, 0, 0, 0, 0, 0, 0, 0, 0]]}

  % Below parameters are not in LCTES2014 paper

  \parameter{\alignedOperands{p}{i}{q}{j}}{whether operands $p$ and $q$ are
    aligned when implemented by instructions $i$ and $j$}
  \json{aligned}{[[69, 17, 71, 17], [70, 24, 71, 24]]}
  \jsonComment{note: the example JSON arrays are extracted from a different
    program (Hexagon programs do not yield alignment constraints)}

  \tableSpace

  \parameter{\alignmentDistance{p}{i}{q}{j}}{alignment distance of operands $p$
    and $q$ when implemented by instructions $i$ and $j$}
  \json{adist}{[0, 1, 1]}
  \jsonComment{note: this parameter is encoded with the same structure as
    \code{aligned}: each aligned operand tuple and its corresponding alignment
    distance are found in the same positions of their respective JSON arrays
    (example: \code{aligned[1]~=~[70, 24, 71, 24]}, \code{adist[1]~=~1}.)}

  \tableSpace

  \parameter{\packedOperands{p}{q}}{whether operands $p$ and $q$ are
    packed}
  \json{packed}{[[13, 14], [34, 35], [54, 55]]}
  \jsonComment{note: the example JSON arrays are extracted from a different
    program (Hexagon programs do not yield packing constraints)}

  \tableSpace

  \parameter{\extensionalOperands{p}{q}}{whether operands $p$ and $q$ are
    related extensionally}
  \json{exrelated}{[[4, 5]]}
  \jsonComment{note: the example JSON arrays are extracted from a different
    program (Hexagon programs do not yield extensional constraints)}

  \tableSpace

  \parameter{\extensionalTable{p}{q}}{table of register assignments for
    operands $p$ and $q$}
  \json{table}{[[0, 1],[2, 3],[4, 5],[6, 7]]}
  \jsonComment{note: the example JSON arrays are extracted from a different
    program (Hexagon programs do not yield extensional constraints)}

  \tableSpace

  \parameter{\activators{o}}{set of instructions that activate operation $o$}
  \json{activators}{[[], [10, 17, 13, 19], [10, 17, 13, 19], \dots, []]}
  \jsonComment{note: the example JSON array is extracted from a different
    program (Hexagon programs do not yet yield activation constraints)}

  \tableSpace

  \parameter{\allAdhocConstraints{}}{set of ad hoc processor constraints}
  \json{E}{[[0, [1, [2, 13], [2, 14]], [2, 15]], \dots]}
  \jsonComment{note: constraints are encoded as trees of expression tuples,
    where the first element of each expression tuple encodes its type as
    follows:}
  \jsonComment{%
    \mbox{%
      \begin{tabular}{| l l |}
        \hline
        \code{0} & \code{xor}\\
        \code{1} & \code{and}\\
        \code{2} & \code{active}\\
        \hline
      \end{tabular}
    }
  }
  \jsonComment{note: the example JSON array is extracted from a different
    program (Hexagon programs do not yet yield ad hoc constraints)}

\end{longtable}
