\begin{longtable}{ l p{12cm} }

  \parameter{B, O, P, T}{sets of blocks, operations, operands and temporaries}
  \json{B}{[0, 1, 2]}
  \json{O}{[0, 1, 2, \dots, 33, 34]}
  \json{P}{[0, 1, 2, \dots, 67, 68]}
  \json{T}{[0, 1, 2, \dots, 31, 32]}

  \tableSpace

  \parameter{\operationBlock{o}}{block to which operation $o$ belongs}
  \json{block}{[0, 0, \dots, 1, 1, 1, 2, 2, 2, 2, 2]}

  \tableSpace

  \parameter{\operands{o}}{set of operands of operation $o$}
  \json{operands}{[[0, 1], [2, 3], [4], \dots, [65, 66], [67], [68]]}

  \tableSpace

  \parameter{\temps{p}}{set of temporaries that can be connected to operand $p$}
  \json{temps}{[[0], [1], [-1, 0], \dots, [30], [29, 31, 32]]}

  \tableSpace

  \parameter{\use{p}}{whether $p$ is a use operand}
  \json{use}{[false, false, true, \dots, true, true]}

  \tableSpace

  \parameter{\adjacent{p}{q}}{whether operands $p$ and $q$ are adjacent}
  \json{adjacent}{[[20, 24], [21, 25], \dots, [60, 61]]}

  \tableSpace

  \parameter
  {\preAssigned{p\hspace{0.025cm}}{r}}
  {whether operand $p$ is preassigned to register $\register{r}$}
  \json{preassign}{[[0, 0], [1, 31], [68, 0]]}

  \tableSpace

  \parameter{\width{t}}{number of register atoms that temporary $t$ occupies}
  \json{width}{[1, 1, 1, \dots, 1, 1]}

  \tableSpace

  \parameter{\frequency{b}}{estimated execution frequency of block $b$}
  \json{freq}{[4, 85, 4]}

  \tableSpace

  % Below parameters are not in LCTES2014 paper

  \parameter{\alignedOperands{p}{i}{q}{j}}{whether operands $p$ and $q$ are
    aligned when implemented by instructions $i$ and $j$}
  \json{aligned}{[[69, 17, 71, 17], [70, 24, 71, 24]]}
  \jsonComment{note: the example JSON arrays are extracted from a different
    program (\code{factorial} does not contain alignment constraints)}

  \tableSpace

  \parameter{\alignmentDistance{p}{i}{q}{j}}{alignment distance of operands $p$
    and $q$ when implemented by instructions $i$ and $j$}
  \json{adist}{[0, 1, 1]}
  \jsonComment{note: this parameter is encoded with the same structure as
    \code{aligned}: each aligned operand tuple and its corresponding alignment
    distance are found in the same positions of their respective JSON arrays
    (example: \code{aligned[1]~=~[70, 24, 71, 24]}, \code{adist[1]~=~1}.)}

  \tableSpace

  \parameter{\packedOperands{p}{q}}{whether operands $p$ and $q$ are
    packed}
  \json{packed}{[[13, 14], [34, 35], [54, 55]]}
  \jsonComment{note: the example JSON arrays are extracted from a different
    program (\code{factorial} does not contain packing constraints)}

  \tableSpace

  \parameter{\minimumLiveDuration{t}}{minimum live duration of temporary $t$ if
    it is live}
  \json{minlive}{[1, 1, 1, \dots, 1, 1]}

  \tableSpace

  \parameter
  {\dependencyGraph{b}}
  {fixed dependency graph of the operations of block $b$}
  \json{dep}{[[[0, 1], \dots, [1, 8], \dots, [10, 11]], [\dots], [\dots]]}

  \tableSpace

  \parameter{\activators{o}}{set of instructions that activate operation $o$}
  \json{activators}{[[], [10, 17, 13, 19], [10, 17, 13, 19], \dots, []]}
  \jsonComment{note: the example JSON array is extracted from a different
    program (Hexagon programs do not yet yield activation constraints)}

  \tableSpace

  \parameter
  {\participative{o}{c}}
    {whether operation $o$ is \emph{participative} with cycle $\cycle{c}$}
  \json{part}{[[2, 3], [1, 1], [60, 24]]}

\end{longtable}
